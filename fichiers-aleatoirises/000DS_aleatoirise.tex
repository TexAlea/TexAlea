\documentclass[a4paper,11pt,fleqn]{article}
\input{preambule}

\setlength{\columnsep}{1cm}
\setlength{\columnseprule}{.5pt}

\newcounter{sujet}



\renewcommand{\headrulewidth}{1pt}
\fancyhead[C]{\textbf{Contrôle \no{5}}} 
\fancyhead[L]{\textbf{}}
\fancyhead[R]{\textbf{Sujet \Alph{sujet}}}

\renewcommand{\footrulewidth}{0pt}
\fancyfoot[C]{} 
\fancyfoot[L]{}
\fancyfoot[R]{}

\begin{document}\stepcounter{sujet}
\exo{ : Compléter (sans utiliser d'arrondis)}

\begin{multicols}{4}
\begin{spacing}{2}
$8\times\ldots\ldots=6$

$6\times\ldots\ldots=5$

$2\times\ldots\ldots=5$

$4\times\ldots\ldots=9$

$6\times\ldots\ldots=7$

$6\times\ldots\ldots=2$

$2\times\ldots\ldots=4$

$3\times\ldots\ldots=4$

\end{spacing}
\end{multicols}

\exo{} % Il faudra aléatoiriser les longueurs

	\begin{center}
	\begin{tikzpicture}[scale=.8]
		\NoAutoSpacing
		\tkzDefPoints{.6/-.38/S,3.6/-.38/O,5.74/3/B}
		\tkzDrawPolygon(S,O,B)
		\tkzMarkAngle[mark=x](O,S,B)
		\tkzMarkAngle[mark=o,size=.5](B,O,S)
		\tkzLabelPoints[left](S)
		\tkzLabelPoints[right](O)
		\tkzLabelPoints[above](B)
		\tkzLabelSegment[below](S,O){4 cm}
		\tkzLabelSegment[below,sloped](O,B){5 cm}
	\end{tikzpicture}
	\hspace{3cm}	
	\begin{tikzpicture}[scale=1.2]
		\NoAutoSpacing
		\tkzDefPoints{.6/-.38/C,3.6/-.38/P,5.74/3/V}
		\tkzDrawPolygon(C,P,V)
		\tkzMarkAngle[mark=x](P,C,V)
		\tkzMarkAngle[mark=o,size=.5](V,P,C)
		\tkzLabelPoints[left](C)
		\tkzLabelPoints[right](P)
		\tkzLabelPoints[above](V)
		\tkzLabelSegment[below](C,P){\nombre{4.8} cm}
		\tkzLabelSegment[above,sloped](C,V){\nombre{8.4} cm}
	\end{tikzpicture}
 	
\end{center}

Calculer $PV$ et $SB$. Donner la valeur exacte ou un arrondi au millimètre près. \textit{Vous veillerez à bien détailler les différentes étapes de votre raisonnement.}

\exo{}
\begin{enumerate}
	\item Un article à 530~€ est soldé à $-60~\%$, quel est son nouveau prix ?
	\item Un salaire de \nombre{2560}~€ augmente de 8~\%, quel est son nouveau montant ?
	\item Une facture d'assurance était à 479~€ en 2015, elle a augmenté de 6~\% en 2016 et de 2~\% en 2017. Quel est son nouveau montant ?
\end{enumerate}

\exo{}
Exprimer les variations de prix suivantes en pourcentage du prix de départ.

\begin{multicols}{4}
20~€ $\longrightarrow$ 32~€

70~€ $\longrightarrow$ 63~€

60~€ $\longrightarrow$ 78~€

78~€ $\longrightarrow$ 60~€
\end{multicols}

\exo{}

\begin{enumerate}
	\item Lors du premier tour d'une élection il y a eu \nombre{5920} votes exprimés. Un candidat a réalisé un score arrondi de \nombre{10.81}~\%.
	
	Combien de personnes ont voté pour lui lors de ce premier tour ? 
	\item Voici les résultats du second tour d'une élection  : 
	\qquad
	\begin{tabular}{|l|c|}
	\hline
	Candidat 1 & \nombre{23462}\\
	\hline
	Candidat 2 & \nombre{23389}\\
	\hline
	Votes blancs et nuls & \nombre{396}\\
	\hline
	Nombre de votants & \nombre{47247}\\
	\hline
	\end{tabular}
	
		\begin{enumerate}
			\item Déterminer le nombre de votes exprimés.
			\item Calculer le pourcentage de votes exprimés pour chacun des deux candidats. Donner un arrondi au centième près.
		\end{enumerate}
\end{enumerate}
%%%%%%%%%%%%%%%%%%%%%%%%%%%%
\newpage
\setcounter{exo}{0}
\setcounter{section}{0}
\stepcounter{sujet}
\exo{ : Compléter (sans utiliser d'arrondis)}

\begin{multicols}{4}
\begin{spacing}{2}
$9\times\ldots\ldots=8$

$7\times\ldots\ldots=9$

$5\times\ldots\ldots=9$

$4\times\ldots\ldots=7$

$8\times\ldots\ldots=2$

$4\times\ldots\ldots=6$

$3\times\ldots\ldots=6$

$5\times\ldots\ldots=3$

\end{spacing}
\end{multicols}

\exo{} % Il faudra aléatoiriser les longueurs

	\begin{center}
	\begin{tikzpicture}[scale=.8]
		\NoAutoSpacing
		\tkzDefPoints{.6/-.38/K,3.6/-.38/E,5.74/3/U}
		\tkzDrawPolygon(K,E,U)
		\tkzMarkAngle[mark=x](E,K,U)
		\tkzMarkAngle[mark=o,size=.5](U,E,K)
		\tkzLabelPoints[left](K)
		\tkzLabelPoints[right](E)
		\tkzLabelPoints[above](U)
		\tkzLabelSegment[below](K,E){4 cm}
		\tkzLabelSegment[below,sloped](E,U){6 cm}
	\end{tikzpicture}
	\hspace{3cm}	
	\begin{tikzpicture}[scale=1.2]
		\NoAutoSpacing
		\tkzDefPoints{.6/-.38/D,3.6/-.38/Z,5.74/3/J}
		\tkzDrawPolygon(D,Z,J)
		\tkzMarkAngle[mark=x](Z,D,J)
		\tkzMarkAngle[mark=o,size=.5](J,Z,D)
		\tkzLabelPoints[left](D)
		\tkzLabelPoints[right](Z)
		\tkzLabelPoints[above](J)
		\tkzLabelSegment[below](D,Z){\nombre{5.2} cm}
		\tkzLabelSegment[above,sloped](D,J){\nombre{10.4} cm}
	\end{tikzpicture}
 	
\end{center}

Calculer $ZJ$ et $KU$. Donner la valeur exacte ou un arrondi au millimètre près. \textit{Vous veillerez à bien détailler les différentes étapes de votre raisonnement.}

\exo{}
\begin{enumerate}
	\item Un article à 420~€ est soldé à $-30~\%$, quel est son nouveau prix ?
	\item Un salaire de \nombre{4360}~€ augmente de 9~\%, quel est son nouveau montant ?
	\item Une facture d'assurance était à 240~€ en 2015, elle a augmenté de 7~\% en 2016 et de 5~\% en 2017. Quel est son nouveau montant ?
\end{enumerate}

\exo{}
Exprimer les variations de prix suivantes en pourcentage du prix de départ.

\begin{multicols}{4}
80~€ $\longrightarrow$ 120~€

80~€ $\longrightarrow$ 32~€

30~€ $\longrightarrow$ 54~€

54~€ $\longrightarrow$ 30~€
\end{multicols}

\exo{}

\begin{enumerate}
	\item Lors du premier tour d'une élection il y a eu \nombre{4156} votes exprimés. Un candidat a réalisé un score arrondi de \nombre{16.72}~\%.
	
	Combien de personnes ont voté pour lui lors de ce premier tour ? 
	\item Voici les résultats du second tour d'une élection  : 
	\qquad
	\begin{tabular}{|l|c|}
	\hline
	Candidat 1 & \nombre{25705}\\
	\hline
	Candidat 2 & \nombre{24426}\\
	\hline
	Votes blancs et nuls & \nombre{226}\\
	\hline
	Nombre de votants & \nombre{50357}\\
	\hline
	\end{tabular}
	
		\begin{enumerate}
			\item Déterminer le nombre de votes exprimés.
			\item Calculer le pourcentage de votes exprimés pour chacun des deux candidats. Donner un arrondi au centième près.
		\end{enumerate}
\end{enumerate}
%%%%%%%%%%%%%%%%%%%%%%%%%%%%
\newpage
\setcounter{exo}{0}
\setcounter{section}{0}
\stepcounter{sujet}
\exo{ : Compléter (sans utiliser d'arrondis)}

\begin{multicols}{4}
\begin{spacing}{2}
$2\times\ldots\ldots=3$

$8\times\ldots\ldots=3$

$3\times\ldots\ldots=9$

$6\times\ldots\ldots=3$

$6\times\ldots\ldots=9$

$3\times\ldots\ldots=9$

$5\times\ldots\ldots=4$

$4\times\ldots\ldots=7$

\end{spacing}
\end{multicols}

\exo{} % Il faudra aléatoiriser les longueurs

	\begin{center}
	\begin{tikzpicture}[scale=.8]
		\NoAutoSpacing
		\tkzDefPoints{.6/-.38/X,3.6/-.38/V,5.74/3/C}
		\tkzDrawPolygon(X,V,C)
		\tkzMarkAngle[mark=x](V,X,C)
		\tkzMarkAngle[mark=o,size=.5](C,V,X)
		\tkzLabelPoints[left](X)
		\tkzLabelPoints[right](V)
		\tkzLabelPoints[above](C)
		\tkzLabelSegment[below](X,V){3 cm}
		\tkzLabelSegment[below,sloped](V,C){5 cm}
	\end{tikzpicture}
	\hspace{3cm}	
	\begin{tikzpicture}[scale=1.2]
		\NoAutoSpacing
		\tkzDefPoints{.6/-.38/B,3.6/-.38/K,5.74/3/F}
		\tkzDrawPolygon(B,K,F)
		\tkzMarkAngle[mark=x](K,B,F)
		\tkzMarkAngle[mark=o,size=.5](F,K,B)
		\tkzLabelPoints[left](B)
		\tkzLabelPoints[right](K)
		\tkzLabelPoints[above](F)
		\tkzLabelSegment[below](B,K){\nombre{4.2} cm}
		\tkzLabelSegment[above,sloped](B,F){\nombre{9.8} cm}
	\end{tikzpicture}
 	
\end{center}

Calculer $KF$ et $XC$. Donner la valeur exacte ou un arrondi au millimètre près. \textit{Vous veillerez à bien détailler les différentes étapes de votre raisonnement.}

\exo{}
\begin{enumerate}
	\item Un article à 680~€ est soldé à $-60~\%$, quel est son nouveau prix ?
	\item Un salaire de \nombre{2200}~€ augmente de 2~\%, quel est son nouveau montant ?
	\item Une facture d'assurance était à 785~€ en 2015, elle a augmenté de 8~\% en 2016 et de 3~\% en 2017. Quel est son nouveau montant ?
\end{enumerate}

\exo{}
Exprimer les variations de prix suivantes en pourcentage du prix de départ.

\begin{multicols}{4}
80~€ $\longrightarrow$ 136~€

50~€ $\longrightarrow$ 20~€

30~€ $\longrightarrow$ 33~€

33~€ $\longrightarrow$ 30~€
\end{multicols}

\exo{}

\begin{enumerate}
	\item Lors du premier tour d'une élection il y a eu \nombre{5699} votes exprimés. Un candidat a réalisé un score arrondi de \nombre{13.6}~\%.
	
	Combien de personnes ont voté pour lui lors de ce premier tour ? 
	\item Voici les résultats du second tour d'une élection  : 
	\qquad
	\begin{tabular}{|l|c|}
	\hline
	Candidat 1 & \nombre{24118}\\
	\hline
	Candidat 2 & \nombre{18770}\\
	\hline
	Votes blancs et nuls & \nombre{496}\\
	\hline
	Nombre de votants & \nombre{43384}\\
	\hline
	\end{tabular}
	
		\begin{enumerate}
			\item Déterminer le nombre de votes exprimés.
			\item Calculer le pourcentage de votes exprimés pour chacun des deux candidats. Donner un arrondi au centième près.
		\end{enumerate}
\end{enumerate}
%%%%%%%%%%%%%%%%%%%%%%%%%%%%
\newpage
\setcounter{exo}{0}
\setcounter{section}{0}
\stepcounter{sujet}
\exo{ : Compléter (sans utiliser d'arrondis)}

\begin{multicols}{4}
\begin{spacing}{2}
$8\times\ldots\ldots=9$

$3\times\ldots\ldots=8$

$8\times\ldots\ldots=6$

$8\times\ldots\ldots=7$

$3\times\ldots\ldots=5$

$4\times\ldots\ldots=5$

$7\times\ldots\ldots=5$

$7\times\ldots\ldots=8$

\end{spacing}
\end{multicols}

\exo{} % Il faudra aléatoiriser les longueurs

	\begin{center}
	\begin{tikzpicture}[scale=.8]
		\NoAutoSpacing
		\tkzDefPoints{.6/-.38/S,3.6/-.38/F,5.74/3/G}
		\tkzDrawPolygon(S,F,G)
		\tkzMarkAngle[mark=x](F,S,G)
		\tkzMarkAngle[mark=o,size=.5](G,F,S)
		\tkzLabelPoints[left](S)
		\tkzLabelPoints[right](F)
		\tkzLabelPoints[above](G)
		\tkzLabelSegment[below](S,F){3 cm}
		\tkzLabelSegment[below,sloped](F,G){6 cm}
	\end{tikzpicture}
	\hspace{3cm}	
	\begin{tikzpicture}[scale=1.2]
		\NoAutoSpacing
		\tkzDefPoints{.6/-.38/O,3.6/-.38/Q,5.74/3/X}
		\tkzDrawPolygon(O,Q,X)
		\tkzMarkAngle[mark=x](Q,O,X)
		\tkzMarkAngle[mark=o,size=.5](X,Q,O)
		\tkzLabelPoints[left](O)
		\tkzLabelPoints[right](Q)
		\tkzLabelPoints[above](X)
		\tkzLabelSegment[below](O,Q){\nombre{3.9} cm}
		\tkzLabelSegment[above,sloped](O,X){\nombre{10.4} cm}
	\end{tikzpicture}
 	
\end{center}

Calculer $QX$ et $SG$. Donner la valeur exacte ou un arrondi au millimètre près. \textit{Vous veillerez à bien détailler les différentes étapes de votre raisonnement.}

\exo{}
\begin{enumerate}
	\item Un article à 910~€ est soldé à $-80~\%$, quel est son nouveau prix ?
	\item Un salaire de \nombre{5640}~€ augmente de 8~\%, quel est son nouveau montant ?
	\item Une facture d'assurance était à 128~€ en 2015, elle a augmenté de 3~\% en 2016 et de 8~\% en 2017. Quel est son nouveau montant ?
\end{enumerate}

\exo{}
Exprimer les variations de prix suivantes en pourcentage du prix de départ.

\begin{multicols}{4}
70~€ $\longrightarrow$ 98~€

20~€ $\longrightarrow$ 16~€

50~€ $\longrightarrow$ 80~€

80~€ $\longrightarrow$ 50~€
\end{multicols}

\exo{}

\begin{enumerate}
	\item Lors du premier tour d'une élection il y a eu \nombre{5952} votes exprimés. Un candidat a réalisé un score arrondi de \nombre{16.28}~\%.
	
	Combien de personnes ont voté pour lui lors de ce premier tour ? 
	\item Voici les résultats du second tour d'une élection  : 
	\qquad
	\begin{tabular}{|l|c|}
	\hline
	Candidat 1 & \nombre{29736}\\
	\hline
	Candidat 2 & \nombre{28733}\\
	\hline
	Votes blancs et nuls & \nombre{229}\\
	\hline
	Nombre de votants & \nombre{58698}\\
	\hline
	\end{tabular}
	
		\begin{enumerate}
			\item Déterminer le nombre de votes exprimés.
			\item Calculer le pourcentage de votes exprimés pour chacun des deux candidats. Donner un arrondi au centième près.
		\end{enumerate}
\end{enumerate}
\end{document}