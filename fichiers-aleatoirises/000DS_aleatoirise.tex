\documentclass[a4paper,11pt,fleqn]{article}
\input{preambule}

\setlength{\columnsep}{1cm}
\setlength{\columnseprule}{.5pt}

\newcounter{sujet}



\renewcommand{\headrulewidth}{1pt}
\fancyhead[C]{\textbf{Contrôle \no{5}}} 
\fancyhead[L]{\textbf{}}
\fancyhead[R]{\textbf{Sujet \Alph{sujet}}}

\renewcommand{\footrulewidth}{0pt}
\fancyfoot[C]{} 
\fancyfoot[L]{}
\fancyfoot[R]{}

\begin{document}\stepcounter{sujet}
\exo{ : Compléter (sans utiliser d'arrondis)}

\begin{multicols}{4}
\begin{spacing}{2}
$5\times\ldots\ldots=9$

$8\times\ldots\ldots=5$

$5\times\ldots\ldots=8$

$2\times\ldots\ldots=6$

$6\times\ldots\ldots=9$

$5\times\ldots\ldots=6$

$2\times\ldots\ldots=9$

$2\times\ldots\ldots=3$

\end{spacing}
\end{multicols}

\exo{} % Il faudra aléatoiriser les longueurs

	\begin{center}
	\begin{tikzpicture}[scale=.8]
		\NoAutoSpacing
		\tkzDefPoints{.6/-.38/C,3.6/-.38/F,5.74/3/K}
		\tkzDrawPolygon(C,F,K)
		\tkzMarkAngle[mark=x](F,C,K)
		\tkzMarkAngle[mark=o,size=.5](K,F,C)
		\tkzLabelPoints[left](C)
		\tkzLabelPoints[right](F)
		\tkzLabelPoints[above](K)
		\tkzLabelSegment[below](C,F){4 cm}
		\tkzLabelSegment[below,sloped](F,K){5 cm}
	\end{tikzpicture}
	\hspace{3cm}	
	\begin{tikzpicture}[scale=1.2]
		\NoAutoSpacing
		\tkzDefPoints{.6/-.38/T,3.6/-.38/A,5.74/3/O}
		\tkzDrawPolygon(T,A,O)
		\tkzMarkAngle[mark=x](A,T,O)
		\tkzMarkAngle[mark=o,size=.5](O,A,T)
		\tkzLabelPoints[left](T)
		\tkzLabelPoints[right](A)
		\tkzLabelPoints[above](O)
		\tkzLabelSegment[below](T,A){\nombre{4.8} cm}
		\tkzLabelSegment[above,sloped](T,O){\nombre{8.4} cm}
	\end{tikzpicture}
 	
\end{center}

Calculer $AO$ et $CK$. Donner la valeur exacte ou un arrondi au millimètre près. \textit{Vous veillerez à bien détailler les différentes étapes de votre raisonnement.}

\exo{}
\begin{enumerate}
	\item Un article à 530~€ est soldé à $-50~\%$, quel est son nouveau prix ?
	\item Un salaire de \nombre{9570}~€ augmente de 5~\%, quel est son nouveau montant ?
	\item Une facture d'assurance était à 643~€ en 2015, elle a augmenté de 8~\% en 2016 et de 5~\% en 2017. Quel est son nouveau montant ?
\end{enumerate}

\exo{}
Exprimer les variations de prix suivantes en pourcentage du prix de départ.

\begin{multicols}{4}
10~€ $\longrightarrow$ 19~€

70~€ $\longrightarrow$ 56~€

10~€ $\longrightarrow$ 19~€

19~€ $\longrightarrow$ 10~€
\end{multicols}

\exo{}

\begin{enumerate}
	\item Lors du premier tour d'une élection il y a eu \nombre{4859} votes exprimés. Un candidat a réalisé un score arrondi de \nombre{16.24}~\%.
	
	Combien de personnes ont voté pour lui lors de ce premier tour ? 
	\item Voici les résultats du second tour d'une élection  : 
	\qquad
	\begin{tabular}{|l|c|}
	\hline
	Candidat 1 & \nombre{16915}\\
	\hline
	Candidat 2 & \nombre{16261}\\
	\hline
	Votes blancs et nuls & \nombre{202}\\
	\hline
	Nombre de votants & \nombre{33378}\\
	\hline
	\end{tabular}
	
		\begin{enumerate}
			\item Déterminer le nombre de votes exprimés.
			\item Calculer le pourcentage de votes exprimés pour chacun des deux candidats. Donner un arrondi au centième près.
		\end{enumerate}
\end{enumerate}
%%%%%%%%%%%%%%%%%%%%%%%%%%%%
\newpage
\setcounter{section}{0}
\stepcounter{sujet}
\exo{ : Compléter (sans utiliser d'arrondis)}

\begin{multicols}{4}
\begin{spacing}{2}
$4\times\ldots\ldots=5$

$6\times\ldots\ldots=5$

$4\times\ldots\ldots=8$

$6\times\ldots\ldots=5$

$6\times\ldots\ldots=3$

$3\times\ldots\ldots=6$

$7\times\ldots\ldots=9$

$4\times\ldots\ldots=7$

\end{spacing}
\end{multicols}

\exo{} % Il faudra aléatoiriser les longueurs

	\begin{center}
	\begin{tikzpicture}[scale=.8]
		\NoAutoSpacing
		\tkzDefPoints{.6/-.38/B,3.6/-.38/W,5.74/3/P}
		\tkzDrawPolygon(B,W,P)
		\tkzMarkAngle[mark=x](W,B,P)
		\tkzMarkAngle[mark=o,size=.5](P,W,B)
		\tkzLabelPoints[left](B)
		\tkzLabelPoints[right](W)
		\tkzLabelPoints[above](P)
		\tkzLabelSegment[below](B,W){4 cm}
		\tkzLabelSegment[below,sloped](W,P){6 cm}
	\end{tikzpicture}
	\hspace{3cm}	
	\begin{tikzpicture}[scale=1.2]
		\NoAutoSpacing
		\tkzDefPoints{.6/-.38/C,3.6/-.38/D,5.74/3/I}
		\tkzDrawPolygon(C,D,I)
		\tkzMarkAngle[mark=x](D,C,I)
		\tkzMarkAngle[mark=o,size=.5](I,D,C)
		\tkzLabelPoints[left](C)
		\tkzLabelPoints[right](D)
		\tkzLabelPoints[above](I)
		\tkzLabelSegment[below](C,D){\nombre{5.2} cm}
		\tkzLabelSegment[above,sloped](C,I){\nombre{10.4} cm}
	\end{tikzpicture}
 	
\end{center}

Calculer $DI$ et $BP$. Donner la valeur exacte ou un arrondi au millimètre près. \textit{Vous veillerez à bien détailler les différentes étapes de votre raisonnement.}

\exo{}
\begin{enumerate}
	\item Un article à 270~€ est soldé à $-20~\%$, quel est son nouveau prix ?
	\item Un salaire de \nombre{2270}~€ augmente de 4~\%, quel est son nouveau montant ?
	\item Une facture d'assurance était à 686~€ en 2015, elle a augmenté de 6~\% en 2016 et de 4~\% en 2017. Quel est son nouveau montant ?
\end{enumerate}

\exo{}
Exprimer les variations de prix suivantes en pourcentage du prix de départ.

\begin{multicols}{4}
20~€ $\longrightarrow$ 38~€

70~€ $\longrightarrow$ 7~€

80~€ $\longrightarrow$ 152~€

152~€ $\longrightarrow$ 80~€
\end{multicols}

\exo{}

\begin{enumerate}
	\item Lors du premier tour d'une élection il y a eu \nombre{4922} votes exprimés. Un candidat a réalisé un score arrondi de \nombre{14.77}~\%.
	
	Combien de personnes ont voté pour lui lors de ce premier tour ? 
	\item Voici les résultats du second tour d'une élection  : 
	\qquad
	\begin{tabular}{|l|c|}
	\hline
	Candidat 1 & \nombre{26326}\\
	\hline
	Candidat 2 & \nombre{17699}\\
	\hline
	Votes blancs et nuls & \nombre{234}\\
	\hline
	Nombre de votants & \nombre{44259}\\
	\hline
	\end{tabular}
	
		\begin{enumerate}
			\item Déterminer le nombre de votes exprimés.
			\item Calculer le pourcentage de votes exprimés pour chacun des deux candidats. Donner un arrondi au centième près.
		\end{enumerate}
\end{enumerate}
%%%%%%%%%%%%%%%%%%%%%%%%%%%%
\newpage
\setcounter{section}{0}
\stepcounter{sujet}
\exo{ : Compléter (sans utiliser d'arrondis)}

\begin{multicols}{4}
\begin{spacing}{2}
$5\times\ldots\ldots=7$

$7\times\ldots\ldots=8$

$8\times\ldots\ldots=7$

$3\times\ldots\ldots=6$

$2\times\ldots\ldots=5$

$6\times\ldots\ldots=8$

$9\times\ldots\ldots=5$

$6\times\ldots\ldots=4$

\end{spacing}
\end{multicols}

\exo{} % Il faudra aléatoiriser les longueurs

	\begin{center}
	\begin{tikzpicture}[scale=.8]
		\NoAutoSpacing
		\tkzDefPoints{.6/-.38/P,3.6/-.38/H,5.74/3/Q}
		\tkzDrawPolygon(P,H,Q)
		\tkzMarkAngle[mark=x](H,P,Q)
		\tkzMarkAngle[mark=o,size=.5](Q,H,P)
		\tkzLabelPoints[left](P)
		\tkzLabelPoints[right](H)
		\tkzLabelPoints[above](Q)
		\tkzLabelSegment[below](P,H){3 cm}
		\tkzLabelSegment[below,sloped](H,Q){5 cm}
	\end{tikzpicture}
	\hspace{3cm}	
	\begin{tikzpicture}[scale=1.2]
		\NoAutoSpacing
		\tkzDefPoints{.6/-.38/Z,3.6/-.38/X,5.74/3/G}
		\tkzDrawPolygon(Z,X,G)
		\tkzMarkAngle[mark=x](X,Z,G)
		\tkzMarkAngle[mark=o,size=.5](G,X,Z)
		\tkzLabelPoints[left](Z)
		\tkzLabelPoints[right](X)
		\tkzLabelPoints[above](G)
		\tkzLabelSegment[below](Z,X){\nombre{4.2} cm}
		\tkzLabelSegment[above,sloped](Z,G){\nombre{9.8} cm}
	\end{tikzpicture}
 	
\end{center}

Calculer $XG$ et $PQ$. Donner la valeur exacte ou un arrondi au millimètre près. \textit{Vous veillerez à bien détailler les différentes étapes de votre raisonnement.}

\exo{}
\begin{enumerate}
	\item Un article à 720~€ est soldé à $-80~\%$, quel est son nouveau prix ?
	\item Un salaire de \nombre{4110}~€ augmente de 5~\%, quel est son nouveau montant ?
	\item Une facture d'assurance était à 172~€ en 2015, elle a augmenté de 7~\% en 2016 et de 8~\% en 2017. Quel est son nouveau montant ?
\end{enumerate}

\exo{}
Exprimer les variations de prix suivantes en pourcentage du prix de départ.

\begin{multicols}{4}
10~€ $\longrightarrow$ 18~€

10~€ $\longrightarrow$ 9~€

70~€ $\longrightarrow$ 119~€

119~€ $\longrightarrow$ 70~€
\end{multicols}

\exo{}

\begin{enumerate}
	\item Lors du premier tour d'une élection il y a eu \nombre{4932} votes exprimés. Un candidat a réalisé un score arrondi de \nombre{13.5}~\%.
	
	Combien de personnes ont voté pour lui lors de ce premier tour ? 
	\item Voici les résultats du second tour d'une élection  : 
	\qquad
	\begin{tabular}{|l|c|}
	\hline
	Candidat 1 & \nombre{16310}\\
	\hline
	Candidat 2 & \nombre{17907}\\
	\hline
	Votes blancs et nuls & \nombre{272}\\
	\hline
	Nombre de votants & \nombre{34489}\\
	\hline
	\end{tabular}
	
		\begin{enumerate}
			\item Déterminer le nombre de votes exprimés.
			\item Calculer le pourcentage de votes exprimés pour chacun des deux candidats. Donner un arrondi au centième près.
		\end{enumerate}
\end{enumerate}
%%%%%%%%%%%%%%%%%%%%%%%%%%%%
\newpage
\setcounter{section}{0}
\stepcounter{sujet}
\exo{ : Compléter (sans utiliser d'arrondis)}

\begin{multicols}{4}
\begin{spacing}{2}
$6\times\ldots\ldots=9$

$3\times\ldots\ldots=7$

$8\times\ldots\ldots=4$

$7\times\ldots\ldots=8$

$6\times\ldots\ldots=8$

$7\times\ldots\ldots=5$

$5\times\ldots\ldots=2$

$2\times\ldots\ldots=6$

\end{spacing}
\end{multicols}

\exo{} % Il faudra aléatoiriser les longueurs

	\begin{center}
	\begin{tikzpicture}[scale=.8]
		\NoAutoSpacing
		\tkzDefPoints{.6/-.38/F,3.6/-.38/G,5.74/3/J}
		\tkzDrawPolygon(F,G,J)
		\tkzMarkAngle[mark=x](G,F,J)
		\tkzMarkAngle[mark=o,size=.5](J,G,F)
		\tkzLabelPoints[left](F)
		\tkzLabelPoints[right](G)
		\tkzLabelPoints[above](J)
		\tkzLabelSegment[below](F,G){3 cm}
		\tkzLabelSegment[below,sloped](G,J){6 cm}
	\end{tikzpicture}
	\hspace{3cm}	
	\begin{tikzpicture}[scale=1.2]
		\NoAutoSpacing
		\tkzDefPoints{.6/-.38/C,3.6/-.38/W,5.74/3/H}
		\tkzDrawPolygon(C,W,H)
		\tkzMarkAngle[mark=x](W,C,H)
		\tkzMarkAngle[mark=o,size=.5](H,W,C)
		\tkzLabelPoints[left](C)
		\tkzLabelPoints[right](W)
		\tkzLabelPoints[above](H)
		\tkzLabelSegment[below](C,W){\nombre{3.9} cm}
		\tkzLabelSegment[above,sloped](C,H){\nombre{10.4} cm}
	\end{tikzpicture}
 	
\end{center}

Calculer $WH$ et $FJ$. Donner la valeur exacte ou un arrondi au millimètre près. \textit{Vous veillerez à bien détailler les différentes étapes de votre raisonnement.}

\exo{}
\begin{enumerate}
	\item Un article à 880~€ est soldé à $-10~\%$, quel est son nouveau prix ?
	\item Un salaire de \nombre{3500}~€ augmente de 6~\%, quel est son nouveau montant ?
	\item Une facture d'assurance était à 939~€ en 2015, elle a augmenté de 3~\% en 2016 et de 8~\% en 2017. Quel est son nouveau montant ?
\end{enumerate}

\exo{}
Exprimer les variations de prix suivantes en pourcentage du prix de départ.

\begin{multicols}{4}
90~€ $\longrightarrow$ 108~€

90~€ $\longrightarrow$ 45~€

50~€ $\longrightarrow$ 55~€

55~€ $\longrightarrow$ 50~€
\end{multicols}

\exo{}

\begin{enumerate}
	\item Lors du premier tour d'une élection il y a eu \nombre{4684} votes exprimés. Un candidat a réalisé un score arrondi de \nombre{17.31}~\%.
	
	Combien de personnes ont voté pour lui lors de ce premier tour ? 
	\item Voici les résultats du second tour d'une élection  : 
	\qquad
	\begin{tabular}{|l|c|}
	\hline
	Candidat 1 & \nombre{20430}\\
	\hline
	Candidat 2 & \nombre{22371}\\
	\hline
	Votes blancs et nuls & \nombre{307}\\
	\hline
	Nombre de votants & \nombre{43108}\\
	\hline
	\end{tabular}
	
		\begin{enumerate}
			\item Déterminer le nombre de votes exprimés.
			\item Calculer le pourcentage de votes exprimés pour chacun des deux candidats. Donner un arrondi au centième près.
		\end{enumerate}
\end{enumerate}
\end{document}