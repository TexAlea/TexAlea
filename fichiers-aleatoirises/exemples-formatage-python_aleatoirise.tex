\documentclass[a4paper,11pt,fleqn]{article}
\input{preambule}
\begin{document}
\pagestyle{empty}


{Enoncé} \hfill {\huge Fiche d'exercices \no 1 : nom1} \hfill {6ème 8}

\section{Somme algébrique}
\begin{itemize}

  \item $4 +3=\ldots$
  \item $-6 +2=\ldots$
  \item $-2 -3=\ldots$
\end{itemize}


\section{Produits de nombres relatifs}
\begin{itemize}

  \item $7\times(-7)=\ldots$
  \item $5\times(-7)=\ldots$
  \item $6\times(-4)=\ldots$
  \item $5\times(-5)=\ldots$
\end{itemize}


\section{Prix soldés}
\begin{itemize}

  \item Un article coûte 6,20~€ et est soldé à $-80~\%$. Quel est son nouveau prix ?
  \item Un article coûte 4,30~€ et est soldé à $-40~\%$. Quel est son nouveau prix ?
  \item Un article coûte 6,40~€ et est soldé à $-80~\%$. Quel est son nouveau prix ?
\end{itemize}
%%%%%%%%%%%%%%%%%%%%%%%%%%%%
\newpage
\setcounter{exo}{0}
\setcounter{section}{0}
{Enoncé} \hfill {\huge Fiche d'exercices \no 2 : nom2} \hfill {6ème 8}

\section{Somme algébrique}
\begin{itemize}

  \item $-7 +5=\ldots$
  \item $-6 -7=\ldots$
  \item $-2 -7=\ldots$
\end{itemize}


\section{Produits de nombres relatifs}
\begin{itemize}

  \item $3\times(-2)=\ldots$
  \item $-4\times2=\ldots$
  \item $-4\times(-2)=\ldots$
  \item $6\times3=\ldots$
\end{itemize}


\section{Prix soldés}
\begin{itemize}

  \item Un article coûte 9,90~€ et est soldé à $-30~\%$. Quel est son nouveau prix ?
  \item Un article coûte 9,10~€ et est soldé à $-50~\%$. Quel est son nouveau prix ?
  \item Un article coûte 2,80~€ et est soldé à $-50~\%$. Quel est son nouveau prix ?
\end{itemize}
%%%%%%%%%%%%%%%%%%%%%%%%%%%%
\newpage
\setcounter{exo}{0}
\setcounter{section}{0}
{Enoncé} \hfill {\huge Fiche d'exercices \no 3 : nom3} \hfill {6ème 8}

\section{Somme algébrique}
\begin{itemize}

  \item $-5 -5=\ldots$
  \item $8 +6=\ldots$
  \item $2 -5=\ldots$
\end{itemize}


\section{Produits de nombres relatifs}
\begin{itemize}

  \item $-3\times(-2)=\ldots$
  \item $-5\times4=\ldots$
  \item $-6\times4=\ldots$
  \item $-6\times4=\ldots$
\end{itemize}


\section{Prix soldés}
\begin{itemize}

  \item Un article coûte 6,40~€ et est soldé à $-40~\%$. Quel est son nouveau prix ?
  \item Un article coûte 0,70~€ et est soldé à $-70~\%$. Quel est son nouveau prix ?
  \item Un article coûte 8,20~€ et est soldé à $-70~\%$. Quel est son nouveau prix ?
\end{itemize}
%%%%%%%%%%%%%%%%%%%%%%%%%%%%
\newpage
\setcounter{exo}{0}
\setcounter{section}{0}
{Enoncé} \hfill {\huge Fiche d'exercices \no 4 : nom4} \hfill {6ème 8}

\section{Somme algébrique}
\begin{itemize}

  \item $9 +3=\ldots$
  \item $4 +4=\ldots$
  \item $6 -9=\ldots$
\end{itemize}


\section{Produits de nombres relatifs}
\begin{itemize}

  \item $-5\times5=\ldots$
  \item $5\times(-3)=\ldots$
  \item $5\times(-9)=\ldots$
  \item $6\times(-6)=\ldots$
\end{itemize}


\section{Prix soldés}
\begin{itemize}

  \item Un article coûte 3,80~€ et est soldé à $-40~\%$. Quel est son nouveau prix ?
  \item Un article coûte 9,30~€ et est soldé à $-10~\%$. Quel est son nouveau prix ?
  \item Un article coûte 6,30~€ et est soldé à $-90~\%$. Quel est son nouveau prix ?
\end{itemize}
\end{document}