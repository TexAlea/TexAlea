\documentclass[a4paper,11pt,fleqn]{article}
\input{preambule}
\begin{document}
\pagestyle{empty}


{Correction} \hfill {\huge Fiche d'exercices \no 1} \hfill {Classe}

\section{Somme algébrique}
\begin{itemize}

  \item $4 +3=7$
  \item $-6 +2=-4$
  \item $-2 -3=-5$
\end{itemize}

\section{Produits de nombres relatifs}
\begin{itemize}

  \item $7\times(-7)=-49$
  \item $5\times(-7)=-35$
  \item $6\times(-4)=-24$
  \item $5\times(-5)=-25$
\end{itemize}

\section{Prix soldés}
\begin{itemize}

  \item Réduction : $6,20\times80\div100=\nombre{4,96}$~€\\
  Nouveau prix : $6,20-4,96=\nombre{1,24}$~€
  \item Réduction : $4,30\times40\div100=\nombre{1,72}$~€\\
  Nouveau prix : $4,30-1,72=\nombre{2,58}$~€
  \item Réduction : $6,40\times80\div100=\nombre{5,12}$~€\\
  Nouveau prix : $6,40-5,12=\nombre{1,28}$~€
\end{itemize}
%%%%%%%%%%%%%%%%%%%%%%%%%%%%
\newpage
\setcounter{exo}{0}
\setcounter{section}{0}
{Correction} \hfill {\huge Fiche d'exercices \no 2} \hfill {Classe}

\section{Somme algébrique}
\begin{itemize}

  \item $-7 +5=-2$
  \item $-6 -7=-13$
  \item $-2 -7=-9$
\end{itemize}

\section{Produits de nombres relatifs}
\begin{itemize}

  \item $3\times(-2)=-6$
  \item $-4\times2=-8$
  \item $-4\times(-2)=8$
  \item $6\times3=18$
\end{itemize}

\section{Prix soldés}
\begin{itemize}

  \item Réduction : $9,90\times30\div100=\nombre{2,97}$~€\\
  Nouveau prix : $9,90-2,97=\nombre{6,93}$~€
  \item Réduction : $9,10\times50\div100=\nombre{4,55}$~€\\
  Nouveau prix : $9,10-4,55=\nombre{4,55}$~€
  \item Réduction : $2,80\times50\div100=\nombre{1,40}$~€\\
  Nouveau prix : $2,80-1,40=\nombre{1,40}$~€
\end{itemize}
%%%%%%%%%%%%%%%%%%%%%%%%%%%%
\newpage
\setcounter{exo}{0}
\setcounter{section}{0}
{Correction} \hfill {\huge Fiche d'exercices \no 3} \hfill {Classe}

\section{Somme algébrique}
\begin{itemize}

  \item $-5 -5=-10$
  \item $8 +6=14$
  \item $2 -5=-3$
\end{itemize}

\section{Produits de nombres relatifs}
\begin{itemize}

  \item $-3\times(-2)=6$
  \item $-5\times4=-20$
  \item $-6\times4=-24$
  \item $-6\times4=-24$
\end{itemize}

\section{Prix soldés}
\begin{itemize}

  \item Réduction : $6,40\times40\div100=\nombre{2,56}$~€\\
  Nouveau prix : $6,40-2,56=\nombre{3,84}$~€
  \item Réduction : $0,70\times70\div100=\nombre{0,49}$~€\\
  Nouveau prix : $0,70-0,49=\nombre{0,21}$~€
  \item Réduction : $8,20\times70\div100=\nombre{5,74}$~€\\
  Nouveau prix : $8,20-5,74=\nombre{2,46}$~€
\end{itemize}
%%%%%%%%%%%%%%%%%%%%%%%%%%%%
\newpage
\setcounter{exo}{0}
\setcounter{section}{0}
{Correction} \hfill {\huge Fiche d'exercices \no 4} \hfill {Classe}

\section{Somme algébrique}
\begin{itemize}

  \item $9 +3=12$
  \item $4 +4=8$
  \item $6 -9=-3$
\end{itemize}

\section{Produits de nombres relatifs}
\begin{itemize}

  \item $-5\times5=-25$
  \item $5\times(-3)=-15$
  \item $5\times(-9)=-45$
  \item $6\times(-6)=-36$
\end{itemize}

\section{Prix soldés}
\begin{itemize}

  \item Réduction : $3,80\times40\div100=\nombre{1,52}$~€\\
  Nouveau prix : $3,80-1,52=\nombre{2,28}$~€
  \item Réduction : $9,30\times10\div100=\nombre{0,93}$~€\\
  Nouveau prix : $9,30-0,93=\nombre{8,37}$~€
  \item Réduction : $6,30\times90\div100=\nombre{5,67}$~€\\
  Nouveau prix : $6,30-5,67=\nombre{0,63}$~€
\end{itemize}
\end{document}