\documentclass[a4paper,11pt,fleqn]{article}
\input{preambule}
\begin{document}
\pagestyle{empty}


\documentclass[a4paper,11pt,fleqn]{article}		% Présentation générale et mise en page
\input{preambule}
\setlength{\columnsep}{1cm}
\setlength{\columnseprule}{.5pt}


\renewcommand{\headrulewidth}{1pt}
\fancyhead[C]{\textbf{Test -- Calcul littéral - version 1}} 
\fancyhead[L]{Nom:\\ Prénom : }
\fancyhead[R]{Cycle 4}

\renewcommand{\footrulewidth}{1pt}
\fancyfoot[C]{} 
\fancyfoot[L]{}
\fancyfoot[R]{2 mars 2017}

\begin{document}

\vfill

{}\hfill {}
\begin{tabular}{|c|c|c|}
\hline 
B20 & Transformer une phrase en expression littérale & \hspace{1cm} \\ 
\hline 
E30 & Réduire une expression littérale &  \\ 
\hline 
E31 & Développer des expressions littérales &  \\
\hline 
E32 & Factoriser des expressions littérales &  \\
\hline 
\end{tabular} 
{}\hfill {}

\vfill

	
\exo{}
Quel résultat donne chacun de ces 2 programmes de calculs lorsqu'on prend $x$ comme nombre de départ ?	

\begin{center}
\fcolorbox{black}{gray!30}{
\begin{minipage}{0.7\linewidth}
Programme 1 : 
\begin{itemize}
	\item Ajouter -7
	\item Diviser par -3
\end{itemize}
\end{minipage}
}

\medskip
\fcolorbox{black}{gray!30}{
\begin{minipage}{0.7\linewidth}
Programme 2 : 
\begin{itemize}
	\item Prendre le double
	\item Soustraire -4
	\item Multiplier par -6
\end{itemize}
\end{minipage}
}
\end{center}

\exo{}

Simplifier les expressions suivantes.

\begin{multicols}{2}
$B=x\times 8 \times x$\\
$A=7 x +3 x$\\

$D=5\times x\times 8\times x$\\
$C=5\times x +8\times x$\\

$E=2 \times x+6\times (-5) +2 x$\\
%$F=7 \times (3y +4)$\\
$G=a\times 1+a\times 2+2 a\times  a$\\
%$H= -9\times x +5 \times x$
\end{multicols}


\exo{}

Développer et réduire les expressions suivantes.

\begin{multicols}{2}
$A=  -3 (x  +3)$\\
$B= -5 ( -2 x    +7)$\\
$C=( -4  +3 y) y$\\
$D= 7  t(  -9 t -3)$
\end{multicols}

\exo{ }

Factoriser les expressions suivantes.

\begin{multicols}{2}
$E=  63 x  -7$\\
$F=  18 x  +6 x$\\
$G=  2 x  -2 x^2$\\
$H=  6 x^2  -8 x^2$
\end{multicols}



%%%%%%%%%%%%%%%%%%%%%%%%%%%%
\newpage
\setcounter{section}{0}
\documentclass[a4paper,11pt,fleqn]{article}		% Présentation générale et mise en page
\input{preambule}
\setlength{\columnsep}{1cm}
\setlength{\columnseprule}{.5pt}


\renewcommand{\headrulewidth}{1pt}
\fancyhead[C]{\textbf{Test -- Calcul littéral - version 2}} 
\fancyhead[L]{Nom:\\ Prénom : }
\fancyhead[R]{Cycle 4}

\renewcommand{\footrulewidth}{1pt}
\fancyfoot[C]{} 
\fancyfoot[L]{}
\fancyfoot[R]{2 mars 2017}

\begin{document}

\vfill

{}\hfill {}
\begin{tabular}{|c|c|c|}
\hline 
B20 & Transformer une phrase en expression littérale & \hspace{1cm} \\ 
\hline 
E30 & Réduire une expression littérale &  \\ 
\hline 
E31 & Développer des expressions littérales &  \\
\hline 
E32 & Factoriser des expressions littérales &  \\
\hline 
\end{tabular} 
{}\hfill {}

\vfill

	
\exo{}
Quel résultat donne chacun de ces 2 programmes de calculs lorsqu'on prend $x$ comme nombre de départ ?	

\begin{center}
\fcolorbox{black}{gray!30}{
\begin{minipage}{0.7\linewidth}
Programme 1 : 
\begin{itemize}
	\item Ajouter 9
	\item Diviser par 8
\end{itemize}
\end{minipage}
}

\medskip
\fcolorbox{black}{gray!30}{
\begin{minipage}{0.7\linewidth}
Programme 2 : 
\begin{itemize}
	\item Prendre le double
	\item Soustraire 7
	\item Multiplier par -8
\end{itemize}
\end{minipage}
}
\end{center}

\exo{}

Simplifier les expressions suivantes.

\begin{multicols}{2}
$B=x\times 6 \times x$\\
$A=3 x +4 x$\\

$D=-9\times x\times 6\times x$\\
$C=-9\times x +6\times x$\\

$E=7 \times x-9\times 5 +8 x$\\
%$F=-6 \times (4y +5)$\\
$G=a\times 2+a\times 1+1 a\times  a$\\
%$H= 8\times x +4 \times x$
\end{multicols}


\exo{}

Développer et réduire les expressions suivantes.

\begin{multicols}{2}
$A=  2 (x  +4)$\\
$B= -7 ( -9 x    +3)$\\
$C=( 5  +6 y) y$\\
$D= 2  t(  8 t +8)$
\end{multicols}

\exo{ }

Factoriser les expressions suivantes.

\begin{multicols}{2}
$E=  3 x  -3$\\
$F=  63 x  -7 x$\\
$G=  -30 x  +6 x^2$\\
$H=  -5 x^2  -7 x^2$
\end{multicols}



%%%%%%%%%%%%%%%%%%%%%%%%%%%%
\newpage
\setcounter{section}{0}
\documentclass[a4paper,11pt,fleqn]{article}		% Présentation générale et mise en page
\input{preambule}
\setlength{\columnsep}{1cm}
\setlength{\columnseprule}{.5pt}


\renewcommand{\headrulewidth}{1pt}
\fancyhead[C]{\textbf{Test -- Calcul littéral - version 3}} 
\fancyhead[L]{Nom:\\ Prénom : }
\fancyhead[R]{Cycle 4}

\renewcommand{\footrulewidth}{1pt}
\fancyfoot[C]{} 
\fancyfoot[L]{}
\fancyfoot[R]{2 mars 2017}

\begin{document}

\vfill

{}\hfill {}
\begin{tabular}{|c|c|c|}
\hline 
B20 & Transformer une phrase en expression littérale & \hspace{1cm} \\ 
\hline 
E30 & Réduire une expression littérale &  \\ 
\hline 
E31 & Développer des expressions littérales &  \\
\hline 
E32 & Factoriser des expressions littérales &  \\
\hline 
\end{tabular} 
{}\hfill {}

\vfill

	
\exo{}
Quel résultat donne chacun de ces 2 programmes de calculs lorsqu'on prend $x$ comme nombre de départ ?	

\begin{center}
\fcolorbox{black}{gray!30}{
\begin{minipage}{0.7\linewidth}
Programme 1 : 
\begin{itemize}
	\item Ajouter -7
	\item Diviser par -3
\end{itemize}
\end{minipage}
}

\medskip
\fcolorbox{black}{gray!30}{
\begin{minipage}{0.7\linewidth}
Programme 2 : 
\begin{itemize}
	\item Prendre le double
	\item Soustraire -4
	\item Multiplier par -6
\end{itemize}
\end{minipage}
}
\end{center}

\exo{}

Simplifier les expressions suivantes.

\begin{multicols}{2}
$B=x\times 8 \times x$\\
$A=7 x +3 x$\\

$D=5\times x\times 8\times x$\\
$C=5\times x +8\times x$\\

$E=2 \times x+6\times (-5) +2 x$\\
%$F=7 \times (3y +4)$\\
$G=a\times 1+a\times 2+2 a\times  a$\\
%$H= -9\times x +5 \times x$
\end{multicols}


\exo{}

Développer et réduire les expressions suivantes.

\begin{multicols}{2}
$A=  -3 (x  +3)$\\
$B= -5 ( -2 x    +7)$\\
$C=( -4  +3 y) y$\\
$D= 7  t(  -9 t -3)$
\end{multicols}

\exo{ }

Factoriser les expressions suivantes.

\begin{multicols}{2}
$E=  63 x  -7$\\
$F=  18 x  +6 x$\\
$G=  2 x  -2 x^2$\\
$H=  6 x^2  -8 x^2$
\end{multicols}



\raggedcolumns
\end{multicols}


\end{document}
%%%%%%%%%%%%%%%%%%%%%%%%%%%%
\newpage
\setcounter{section}{0}
\documentclass[a4paper,11pt,fleqn]{article}		% Présentation générale et mise en page
\input{preambule}
\setlength{\columnsep}{1cm}
\setlength{\columnseprule}{.5pt}


\renewcommand{\headrulewidth}{1pt}
\fancyhead[C]{\textbf{Test -- Calcul littéral - version 4}} 
\fancyhead[L]{Nom:\\ Prénom : }
\fancyhead[R]{Cycle 4}

\renewcommand{\footrulewidth}{1pt}
\fancyfoot[C]{} 
\fancyfoot[L]{}
\fancyfoot[R]{2 mars 2017}

\begin{document}

\vfill

{}\hfill {}
\begin{tabular}{|c|c|c|}
\hline 
B20 & Transformer une phrase en expression littérale & \hspace{1cm} \\ 
\hline 
E30 & Réduire une expression littérale &  \\ 
\hline 
E31 & Développer des expressions littérales &  \\
\hline 
E32 & Factoriser des expressions littérales &  \\
\hline 
\end{tabular} 
{}\hfill {}

\vfill

	
\exo{}
Quel résultat donne chacun de ces 2 programmes de calculs lorsqu'on prend $x$ comme nombre de départ ?	

\begin{center}
\fcolorbox{black}{gray!30}{
\begin{minipage}{0.7\linewidth}
Programme 1 : 
\begin{itemize}
	\item Ajouter 9
	\item Diviser par 8
\end{itemize}
\end{minipage}
}

\medskip
\fcolorbox{black}{gray!30}{
\begin{minipage}{0.7\linewidth}
Programme 2 : 
\begin{itemize}
	\item Prendre le double
	\item Soustraire 7
	\item Multiplier par -8
\end{itemize}
\end{minipage}
}
\end{center}

\exo{}

Simplifier les expressions suivantes.

\begin{multicols}{2}
$B=x\times 6 \times x$\\
$A=3 x +4 x$\\

$D=-9\times x\times 6\times x$\\
$C=-9\times x +6\times x$\\

$E=7 \times x-9\times 5 +8 x$\\
%$F=-6 \times (4y +5)$\\
$G=a\times 2+a\times 1+1 a\times  a$\\
%$H= 8\times x +4 \times x$
\end{multicols}


\exo{}

Développer et réduire les expressions suivantes.

\begin{multicols}{2}
$A=  2 (x  +4)$\\
$B= -7 ( -9 x    +3)$\\
$C=( 5  +6 y) y$\\
$D= 2  t(  8 t +8)$
\end{multicols}

\exo{ }

Factoriser les expressions suivantes.

\begin{multicols}{2}
$E=  3 x  -3$\\
$F=  63 x  -7 x$\\
$G=  -30 x  +6 x^2$\\
$H=  -5 x^2  -7 x^2$
\end{multicols}



\raggedcolumns
\end{multicols}


\end{document}
\end{document}