
\begin{minipage}{0.15\linewidth}
Version <<version>>
\end{minipage}\hfill
\begin{minipage}{0.75\linewidth}
\begin{center}
	\textsc{\Large{Représenter des nombres\\ sur un axe gradué...}}
\end{center}
\end{minipage}\hfill
\begin{minipage}{0.1\linewidth}
{} \hfill 6$^e$
\end{minipage}

\medskip

\textit{Dans toute cette fiche, sur chaque première demi-droite graduée, on a effectué deux zooms successifs qui sont représentés juste en dessous.}

\vfill

\exon{(Représenter un nombre décimal \quad \textbf{C14})}



\textbf{1.} Place, le plus précisément possible, le point $<<L1[1]>>$ d'abscisse $<<NNN[1]/100>>$ \textbf{sur chacun des trois axes} ci-dessous.

\vfill

{} \hfill 
\axesZoom{<<NNN[1]/100>>}{<<L1[1]>>}{0}{false}{true}
\hfill {}

\vfill

\textbf{2.} Place, le plus précisément possible, le point $<<L2[1]>> (<<NNN[2]/100>>)$ \textbf{sur chacun des trois axes} ci-dessous.

\vfill

{} \hfill 
\axesZoom{<<NNN[2]/100>>}{<<L2[1]>>}{0}{false}{true}
\hfill {}

\vfill

\textbf{3.} Place, le plus précisément possible, le point $<<L3[1]>>$ tel que $x_{<<L3[1]>>}=<<NNN[3]/100>>$ sur chacun des axes ci-dessous.

\vfill

{} \hfill 
\axesZoom{<<NNN[3]/100>>}{<<L3[1]>>}{0}{false}{true}
\hfill {}

\vfill

% La suite ne peut être aléatoirisée qu'à condition d'améliorer les commandes \axesZoom afin de les rendre plus génériques...

%\exon{(Déterminer l'abscisse d'un point \quad \textbf{C14})}
%
%
%Détermine l'abscisse des points $<<L4[1]>>$, $<<L5[1]>>$ et $<<L6[1]>>$ ci-dessous.
%
%
%{} \hfill 
%\axesZoomBis{1.52}{$<<L6[1]>>$}{0}{false}{false}
%\hfill {}
%
%\vfill
%
%\hrule
%
%\vfill
%
%\textbf{Pour les plus rapides :}  détermine les abscisses des points $G$, $H$, $I$, $K$, $L$, $M$, $N$, $P$.
%
%{} \hfill 
%\axesZoomTer{10.31}{$K$}{3}{false}{false}
%\hfill {}
%
%\vfill
%
%{}
%
%\vfill
%
%{} \hfill 
%\axesZoomTerBis{19.99}{$P$}{17}{false}{false}
%\hfill {}

\vfill
 
