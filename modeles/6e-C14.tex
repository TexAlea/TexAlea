%\large

{}

\vspace{-0.8cm}

\textit{Dans toute cette fiche, sur chaque première demi-droite graduée, on a effectué deux zooms successifs qui sont représentés juste en dessous.}

\vfill

\exon{(Représenter un nombre décimal \quad \textbf{C14})}



\textbf{1.} Place, le plus précisément possible, le point $<<L1[1]>>$ d'abscisse $<<NNN[1]/100>>$ \textbf{sur chacun des trois axes} ci-dessous.

\vfill

{} \hfill 
\axesZoom{<<NNN[1]/100>>}{<<L1[1]>>}{0}{false}{true}
\hfill {}

\vfill

\textbf{2.} Place, le plus précisément possible, le point $<<L2[1]>> (<<NNN[2]/100>>)$ \textbf{sur chacun des trois axes} ci-dessous.

\vfill

{} \hfill 
\axesZoom{<<NNN[2]/100>>}{<<L2[1]>>}{0}{false}{true}
\hfill {}

\vfill

\textbf{3.} Place, le plus précisément possible, le point $<<L3[1]>>$ tel que $x_{<<L3[1]>>}=<<NNN[3]/100>>$ sur chacun des axes ci-dessous.

\vfill

{} \hfill 
\axesZoom{<<NNN[3]/100>>}{<<L3[1]>>}{0}{false}{true}
\hfill {}

\vfill


\exon{(Déterminer l'abscisse d'un point \quad \textbf{C14})}


Détermine l'abscisse des points $D$, $E$ et $F$ ci-dessous.


{} \hfill 
\axesZoomBis{1.52}{$F$}{0}{false}{false}
\hfill {}

\vfill

\hrule

\vfill

\textbf{Pour les plus rapides :}  détermine les abscisses des points $G$, $H$, $I$, $K$, $L$, $M$, $N$, $P$.

{} \hfill 
\axesZoomTer{10.31}{$K$}{3}{false}{false}
\hfill {}

\vfill

{}

\vfill

{} \hfill 
\axesZoomTerBis{19.99}{$P$}{17}{false}{false}
\hfill {}

\vfill
 
