Range les nombres ci-dessous du plus grand au plus petit.\\ \\


% Création des variables
% <<affectealeadecimal('a')>>   
% <<affecte(10*var['a']+alea(1,2),'b')>>    
% <<affecte(var['a']+alea(-3,-1),'c')>>
% <<affecte(var['a']+alea(1,4),'d')>>


<<melanger(('La variable a est '~var['a']~'\\\\','La variable b est '~var['b']~'\\\\'))>>

  


$<<ecriture_decimale(var['a'],alea(1,3))>>$ \hfill $<<ecriture_decimale(var['b'],alea(1,3))>>$ \hfill $<<ecriture_decimale(var['c'],alea(1,3))>>$ \hfill $<<ecriture_decimale(var['d'],alea(1,3))>>$


% Création des variables
% <<alea(101,9999,'r')>>   
% <<affecte(var['r']+alea(-3,-1),'s')>>    
% <<affecte(var['r']+alea(1,2),'t')>>
% <<affecte(var['r']+alea(3,4),'u')>>

\bigskip
Range les nombres ci-dessous du plus grand au plus petit.\\ \\



$<<ecriture_decimale(var['r']/DCM[1],alea(1,3))>>$ \hfill $<<ecriture_decimale(var['s']/DCM[2],alea(1,3))>>$ \hfill $<<ecriture_decimale(var['t']/DCM[2],alea(1,3))>>$ \hfill $<<ecriture_decimale(var['u']/DCM[1],alea(1,3))>>$

