\documentclass[a4paper,11pt,fleqn]{article}		% Présentation générale et mise en page


\input{preambule}
\input{preambule-partages}
\input{preambule-axes-gradues}


%\input{preambule}
\setlength{\columnsep}{1cm}
\setlength{\columnseprule}{.5pt}


\renewcommand{\headrulewidth}{1pt}
\fancyhead[C]{\textbf{Entraînement -- Fractions et rationnels niveau 1 (v1)}} 
\fancyhead[L]{}
\fancyhead[R]{Cycle 4}

\renewcommand{\footrulewidth}{1pt}
\fancyfoot[C]{} 
\fancyfoot[L]{}
\fancyfoot[R]{Fractions et rationnels niveau 1 (v1)}

\begin{document}{Enoncé} \hfill {\huge Fiche d'exercices \no 1} \hfill {Classe}

\section{Somme algébrique}
\begin{itemize}

  \item $-7 +4=\ldots$
  \item $2 -8=\ldots$
  \item $-8 +9=\ldots$
\end{itemize}


\section{Produits de nombres relatifs}
\begin{itemize}

  \item $8\times(-3)=\ldots$
  \item $9\times(-8)=\ldots$
  \item $6\times(-3)=\ldots$
  \item $3\times(-8)=\ldots$
\end{itemize}


\section{Prix soldés}
\begin{itemize}

  \item Un article coûte 9,90~€ et est soldé à $-40~\%$. Quel est son nouveau prix ?
  \item Un article coûte 8,60~€ et est soldé à $-50~\%$. Quel est son nouveau prix ?
  \item Un article coûte 4,20~€ et est soldé à $-10~\%$. Quel est son nouveau prix ?
\end{itemize}
%%%%%%%%%%%%%%%%%%%%%%%%%%%%
\newpage
\setcounter{exo}{0}
\setcounter{section}{0}
{Enoncé} \hfill {\huge Fiche d'exercices \no 2} \hfill {Classe}

\section{Somme algébrique}
\begin{itemize}

  \item $3 +2=\ldots$
  \item $8 +5=\ldots$
  \item $8 +2=\ldots$
\end{itemize}


\section{Produits de nombres relatifs}
\begin{itemize}

  \item $4\times(-7)=\ldots$
  \item $-3\times4=\ldots$
  \item $8\times(-6)=\ldots$
  \item $7\times2=\ldots$
\end{itemize}


\section{Prix soldés}
\begin{itemize}

  \item Un article coûte 3,20~€ et est soldé à $-90~\%$. Quel est son nouveau prix ?
  \item Un article coûte 5,80~€ et est soldé à $-50~\%$. Quel est son nouveau prix ?
  \item Un article coûte 4,30~€ et est soldé à $-80~\%$. Quel est son nouveau prix ?
\end{itemize}
%%%%%%%%%%%%%%%%%%%%%%%%%%%%
\newpage
\setcounter{exo}{0}
\setcounter{section}{0}
{Enoncé} \hfill {\huge Fiche d'exercices \no 3} \hfill {Classe}

\section{Somme algébrique}
\begin{itemize}

  \item $-8 -4=\ldots$
  \item $-3 +5=\ldots$
  \item $9 -8=\ldots$
\end{itemize}


\section{Produits de nombres relatifs}
\begin{itemize}

  \item $-6\times(-7)=\ldots$
  \item $2\times(-9)=\ldots$
  \item $-5\times(-3)=\ldots$
  \item $-6\times(-3)=\ldots$
\end{itemize}


\section{Prix soldés}
\begin{itemize}

  \item Un article coûte 9,60~€ et est soldé à $-20~\%$. Quel est son nouveau prix ?
  \item Un article coûte 1,60~€ et est soldé à $-50~\%$. Quel est son nouveau prix ?
  \item Un article coûte 8,80~€ et est soldé à $-90~\%$. Quel est son nouveau prix ?
\end{itemize}
%%%%%%%%%%%%%%%%%%%%%%%%%%%%
\newpage
\setcounter{exo}{0}
\setcounter{section}{0}
{Enoncé} \hfill {\huge Fiche d'exercices \no 4} \hfill {Classe}

\section{Somme algébrique}
\begin{itemize}

  \item $-4 +9=\ldots$
  \item $-6 -7=\ldots$
  \item $-4 -9=\ldots$
\end{itemize}


\section{Produits de nombres relatifs}
\begin{itemize}

  \item $9\times(-9)=\ldots$
  \item $-2\times5=\ldots$
  \item $4\times5=\ldots$
  \item $6\times5=\ldots$
\end{itemize}


\section{Prix soldés}
\begin{itemize}

  \item Un article coûte 5,30~€ et est soldé à $-70~\%$. Quel est son nouveau prix ?
  \item Un article coûte 0,10~€ et est soldé à $-10~\%$. Quel est son nouveau prix ?
  \item Un article coûte 1.0~€ et est soldé à $-10~\%$. Quel est son nouveau prix ?
\end{itemize}
%%%%%%%%%%%%%%%%%%%%%%%%%%%%
\newpage
\setcounter{exo}{0}
\setcounter{section}{0}
{Enoncé} \hfill {\huge Fiche d'exercices \no 5} \hfill {Classe}

\section{Somme algébrique}
\begin{itemize}

  \item $7 -8=\ldots$
  \item $4 -8=\ldots$
  \item $-9 +2=\ldots$
\end{itemize}


\section{Produits de nombres relatifs}
\begin{itemize}

  \item $2\times7=\ldots$
  \item $4\times2=\ldots$
  \item $-5\times9=\ldots$
  \item $5\times(-7)=\ldots$
\end{itemize}


\section{Prix soldés}
\begin{itemize}

  \item Un article coûte 4,70~€ et est soldé à $-10~\%$. Quel est son nouveau prix ?
  \item Un article coûte 8,30~€ et est soldé à $-80~\%$. Quel est son nouveau prix ?
  \item Un article coûte 6,40~€ et est soldé à $-20~\%$. Quel est son nouveau prix ?
\end{itemize}
%%%%%%%%%%%%%%%%%%%%%%%%%%%%
\newpage
\setcounter{exo}{0}
\setcounter{section}{0}
{Enoncé} \hfill {\huge Fiche d'exercices \no 6} \hfill {Classe}

\section{Somme algébrique}
\begin{itemize}

  \item $2 +9=\ldots$
  \item $-5 +9=\ldots$
  \item $9 -9=\ldots$
\end{itemize}


\section{Produits de nombres relatifs}
\begin{itemize}

  \item $3\times(-3)=\ldots$
  \item $-4\times(-8)=\ldots$
  \item $-3\times3=\ldots$
  \item $5\times2=\ldots$
\end{itemize}


\section{Prix soldés}
\begin{itemize}

  \item Un article coûte 2,50~€ et est soldé à $-10~\%$. Quel est son nouveau prix ?
  \item Un article coûte 8,90~€ et est soldé à $-70~\%$. Quel est son nouveau prix ?
  \item Un article coûte 9,40~€ et est soldé à $-30~\%$. Quel est son nouveau prix ?
\end{itemize}
\end{document}