\documentclass[a4paper,11pt,fleqn]{article}
\input{preambule}
\begin{document}
\pagestyle{empty}


{Enoncé} \hfill {\huge Fiche d'exercices \no 1} \hfill {Classe}

\section{Somme algébrique}
\begin{itemize}

  \item $6 +3=\ldots$
  \item $-6 -7=\ldots$
  \item $3 -7=\ldots$
\end{itemize}


\section{Produits de nombres relatifs}
\begin{itemize}

  \item $5\times6=\ldots$
  \item $2\times7=\ldots$
  \item $6\times2=\ldots$
  \item $9\times(-3)=\ldots$
\end{itemize}


\section{Prix soldés}
\begin{itemize}

  \item Un article coûte 5,40~€ et est soldé à $-80~\%$. Quel est son nouveau prix ?
  \item Un article coûte 2.0~€ et est soldé à $-30~\%$. Quel est son nouveau prix ?
  \item Un article coûte 1,20~€ et est soldé à $-80~\%$. Quel est son nouveau prix ?
\end{itemize}
%%%%%%%%%%%%%%%%%%%%%%%%%%%%
\newpage
\setcounter{exo}{0}
\setcounter{section}{0}
{Enoncé} \hfill {\huge Fiche d'exercices \no 2} \hfill {Classe}

\section{Somme algébrique}
\begin{itemize}

  \item $-9 -7=\ldots$
  \item $5 -9=\ldots$
  \item $-5 +2=\ldots$
\end{itemize}


\section{Produits de nombres relatifs}
\begin{itemize}

  \item $-4\times(-2)=\ldots$
  \item $-4\times3=\ldots$
  \item $-5\times2=\ldots$
  \item $-6\times(-6)=\ldots$
\end{itemize}


\section{Prix soldés}
\begin{itemize}

  \item Un article coûte 2,70~€ et est soldé à $-30~\%$. Quel est son nouveau prix ?
  \item Un article coûte 5,50~€ et est soldé à $-20~\%$. Quel est son nouveau prix ?
  \item Un article coûte 8,30~€ et est soldé à $-50~\%$. Quel est son nouveau prix ?
\end{itemize}
%%%%%%%%%%%%%%%%%%%%%%%%%%%%
\newpage
\setcounter{exo}{0}
\setcounter{section}{0}
{Enoncé} \hfill {\huge Fiche d'exercices \no 3} \hfill {Classe}

\section{Somme algébrique}
\begin{itemize}

  \item $-4 -7=\ldots$
  \item $8 +8=\ldots$
  \item $5 -3=\ldots$
\end{itemize}


\section{Produits de nombres relatifs}
\begin{itemize}

  \item $8\times5=\ldots$
  \item $7\times2=\ldots$
  \item $-6\times(-7)=\ldots$
  \item $9\times5=\ldots$
\end{itemize}


\section{Prix soldés}
\begin{itemize}

  \item Un article coûte 4,10~€ et est soldé à $-60~\%$. Quel est son nouveau prix ?
  \item Un article coûte 9.0~€ et est soldé à $-70~\%$. Quel est son nouveau prix ?
  \item Un article coûte 2,50~€ et est soldé à $-80~\%$. Quel est son nouveau prix ?
\end{itemize}
%%%%%%%%%%%%%%%%%%%%%%%%%%%%
\newpage
\setcounter{exo}{0}
\setcounter{section}{0}
{Enoncé} \hfill {\huge Fiche d'exercices \no 4} \hfill {Classe}

\section{Somme algébrique}
\begin{itemize}

  \item $9 -9=\ldots$
  \item $3 +8=\ldots$
  \item $3 +3=\ldots$
\end{itemize}


\section{Produits de nombres relatifs}
\begin{itemize}

  \item $3\times5=\ldots$
  \item $-7\times(-9)=\ldots$
  \item $3\times2=\ldots$
  \item $-3\times2=\ldots$
\end{itemize}


\section{Prix soldés}
\begin{itemize}

  \item Un article coûte 7,10~€ et est soldé à $-30~\%$. Quel est son nouveau prix ?
  \item Un article coûte 6.0~€ et est soldé à $-70~\%$. Quel est son nouveau prix ?
  \item Un article coûte 7,70~€ et est soldé à $-50~\%$. Quel est son nouveau prix ?
\end{itemize}
%%%%%%%%%%%%%%%%%%%%%%%%%%%%
\newpage
\setcounter{exo}{0}
\setcounter{section}{0}
{Enoncé} \hfill {\huge Fiche d'exercices \no 5} \hfill {Classe}

\section{Somme algébrique}
\begin{itemize}

  \item $3 +8=\ldots$
  \item $6 +4=\ldots$
  \item $-4 -9=\ldots$
\end{itemize}


\section{Produits de nombres relatifs}
\begin{itemize}

  \item $8\times(-6)=\ldots$
  \item $-4\times3=\ldots$
  \item $-6\times9=\ldots$
  \item $-5\times(-3)=\ldots$
\end{itemize}


\section{Prix soldés}
\begin{itemize}

  \item Un article coûte 7,60~€ et est soldé à $-70~\%$. Quel est son nouveau prix ?
  \item Un article coûte 5,60~€ et est soldé à $-60~\%$. Quel est son nouveau prix ?
  \item Un article coûte 9,70~€ et est soldé à $-40~\%$. Quel est son nouveau prix ?
\end{itemize}
%%%%%%%%%%%%%%%%%%%%%%%%%%%%
\newpage
\setcounter{exo}{0}
\setcounter{section}{0}
{Enoncé} \hfill {\huge Fiche d'exercices \no 6} \hfill {Classe}

\section{Somme algébrique}
\begin{itemize}

  \item $5 -7=\ldots$
  \item $5 -4=\ldots$
  \item $-4 +6=\ldots$
\end{itemize}


\section{Produits de nombres relatifs}
\begin{itemize}

  \item $3\times(-4)=\ldots$
  \item $2\times2=\ldots$
  \item $-8\times(-2)=\ldots$
  \item $-9\times(-9)=\ldots$
\end{itemize}


\section{Prix soldés}
\begin{itemize}

  \item Un article coûte 4,60~€ et est soldé à $-70~\%$. Quel est son nouveau prix ?
  \item Un article coûte 5,70~€ et est soldé à $-50~\%$. Quel est son nouveau prix ?
  \item Un article coûte 7,40~€ et est soldé à $-10~\%$. Quel est son nouveau prix ?
\end{itemize}
%%%%%%%%%%%%%%%%%%%%%%%%%%%%
\newpage
\setcounter{exo}{0}
\setcounter{section}{0}
{Enoncé} \hfill {\huge Fiche d'exercices \no 7} \hfill {Classe}

\section{Somme algébrique}
\begin{itemize}

  \item $6 +7=\ldots$
  \item $-3 -6=\ldots$
  \item $4 +3=\ldots$
\end{itemize}


\section{Produits de nombres relatifs}
\begin{itemize}

  \item $2\times5=\ldots$
  \item $-7\times6=\ldots$
  \item $9\times(-3)=\ldots$
  \item $-2\times(-3)=\ldots$
\end{itemize}


\section{Prix soldés}
\begin{itemize}

  \item Un article coûte 4,20~€ et est soldé à $-10~\%$. Quel est son nouveau prix ?
  \item Un article coûte 0,80~€ et est soldé à $-20~\%$. Quel est son nouveau prix ?
  \item Un article coûte 3,20~€ et est soldé à $-20~\%$. Quel est son nouveau prix ?
\end{itemize}
%%%%%%%%%%%%%%%%%%%%%%%%%%%%
\newpage
\setcounter{exo}{0}
\setcounter{section}{0}
{Enoncé} \hfill {\huge Fiche d'exercices \no 8} \hfill {Classe}

\section{Somme algébrique}
\begin{itemize}

  \item $-8 -4=\ldots$
  \item $8 -6=\ldots$
  \item $-4 -9=\ldots$
\end{itemize}


\section{Produits de nombres relatifs}
\begin{itemize}

  \item $-2\times(-6)=\ldots$
  \item $-8\times(-5)=\ldots$
  \item $-9\times7=\ldots$
  \item $-7\times(-7)=\ldots$
\end{itemize}


\section{Prix soldés}
\begin{itemize}

  \item Un article coûte 8,20~€ et est soldé à $-30~\%$. Quel est son nouveau prix ?
  \item Un article coûte 2,90~€ et est soldé à $-30~\%$. Quel est son nouveau prix ?
  \item Un article coûte 1,40~€ et est soldé à $-20~\%$. Quel est son nouveau prix ?
\end{itemize}
%%%%%%%%%%%%%%%%%%%%%%%%%%%%
\newpage
\setcounter{exo}{0}
\setcounter{section}{0}
{Enoncé} \hfill {\huge Fiche d'exercices \no 9} \hfill {Classe}

\section{Somme algébrique}
\begin{itemize}

  \item $-4 -3=\ldots$
  \item $5 +7=\ldots$
  \item $2 +9=\ldots$
\end{itemize}


\section{Produits de nombres relatifs}
\begin{itemize}

  \item $-6\times(-9)=\ldots$
  \item $2\times(-3)=\ldots$
  \item $7\times9=\ldots$
  \item $-4\times8=\ldots$
\end{itemize}


\section{Prix soldés}
\begin{itemize}

  \item Un article coûte 8,60~€ et est soldé à $-90~\%$. Quel est son nouveau prix ?
  \item Un article coûte 7,10~€ et est soldé à $-70~\%$. Quel est son nouveau prix ?
  \item Un article coûte 6,20~€ et est soldé à $-70~\%$. Quel est son nouveau prix ?
\end{itemize}
\end{document}