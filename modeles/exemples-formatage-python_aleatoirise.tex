\documentclass[a4paper,11pt,fleqn]{article}
\input{preambule}
\begin{document}
\pagestyle{empty}


{Enoncé} \hfill {\huge Fiche d'exercices \no 1} \hfill {Classe}

\section{Somme algébrique}
\begin{itemize}

  \item $2 +7=\ldots$
  \item $5 -6=\ldots$
  \item $7 -2=\ldots$
\end{itemize}


\section{Produits de nombres relatifs}
\begin{itemize}

  \item $3\times(-5)=\ldots$
  \item $-9\times3=\ldots$
  \item $7\times3=\ldots$
  \item $-3\times(-7)=\ldots$
\end{itemize}


\section{Prix soldés}
\begin{itemize}

  \item Un article coûte 4,90~€ et est soldé à $-40~\%$. Quel est son nouveau prix ?
  \item Un article coûte 5,40~€ et est soldé à $-30~\%$. Quel est son nouveau prix ?
  \item Un article coûte 3,60~€ et est soldé à $-10~\%$. Quel est son nouveau prix ?
\end{itemize}
%%%%%%%%%%%%%%%%%%%%%%%%%%%%
\newpage
\setcounter{exo}{0}
\setcounter{section}{0}
{Enoncé} \hfill {\huge Fiche d'exercices \no 2} \hfill {Classe}

\section{Somme algébrique}
\begin{itemize}

  \item $4 -7=\ldots$
  \item $5 +6=\ldots$
  \item $8 -9=\ldots$
\end{itemize}


\section{Produits de nombres relatifs}
\begin{itemize}

  \item $7\times6=\ldots$
  \item $7\times(-7)=\ldots$
  \item $8\times3=\ldots$
  \item $-4\times4=\ldots$
\end{itemize}


\section{Prix soldés}
\begin{itemize}

  \item Un article coûte 4.0~€ et est soldé à $-70~\%$. Quel est son nouveau prix ?
  \item Un article coûte 2,30~€ et est soldé à $-40~\%$. Quel est son nouveau prix ?
  \item Un article coûte 0,90~€ et est soldé à $-70~\%$. Quel est son nouveau prix ?
\end{itemize}
\end{document}