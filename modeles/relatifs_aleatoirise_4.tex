\documentclass[a4paper,11pt,fleqn]{article}
\input{preambule}
\begin{document}
\pagestyle{empty}


\section{Somme algébrique}
\begin{itemize}

3
3
3
3
3
3
3
3
3
3
  \item $3+6=\ldots$
  \item $8+8=\ldots$
  \item $4+7=\ldots$
\end{itemize}

\section{Correction -- Somme algébrique}
\begin{itemize}

  \item $3+6=9$
  \item $8+8=16$
  \item $4+7=11$
\end{itemize}

\section{Produits de nombres relatifs}
\begin{itemize}

  \item $9\times7=\ldots$
  \item $2\times3=\ldots$
  \item $3\times4=\ldots$
  \item $6\times7=\ldots$
\end{itemize}

\section{Correction -- Produits de nombres relatifs}
\begin{itemize}

  \item $9\times7=63$
  \item $2\times3=6$
  \item $3\times4=12$
  \item $6\times7=42$
\end{itemize}
%%%%%%%%%%%%%%%%%%%%%%%%%%%%
\newpage
\setcounter{exo}{0}
\setcounter{section}{0}
\section{Somme algébrique}
\begin{itemize}

3
3
3
3
3
3
3
3
3
3
  \item $5+4=\ldots$
  \item $4+8=\ldots$
  \item $3+5=\ldots$
\end{itemize}

\section{Correction -- Somme algébrique}
\begin{itemize}

  \item $5+4=9$
  \item $4+8=12$
  \item $3+5=8$
\end{itemize}

\section{Produits de nombres relatifs}
\begin{itemize}

  \item $3\times3=\ldots$
  \item $2\times3=\ldots$
  \item $3\times2=\ldots$
  \item $3\times4=\ldots$
\end{itemize}

\section{Correction -- Produits de nombres relatifs}
\begin{itemize}

  \item $3\times3=9$
  \item $2\times3=6$
  \item $3\times2=6$
  \item $3\times4=12$
\end{itemize}
%%%%%%%%%%%%%%%%%%%%%%%%%%%%
\newpage
\setcounter{exo}{0}
\setcounter{section}{0}
\section{Somme algébrique}
\begin{itemize}

3
3
3
3
3
3
3
3
3
3
  \item $6+3=\ldots$
  \item $7+6=\ldots$
  \item $9+6=\ldots$
\end{itemize}

\section{Correction -- Somme algébrique}
\begin{itemize}

  \item $6+3=9$
  \item $7+6=13$
  \item $9+6=15$
\end{itemize}

\section{Produits de nombres relatifs}
\begin{itemize}

  \item $6\times9=\ldots$
  \item $3\times9=\ldots$
  \item $4\times8=\ldots$
  \item $6\times2=\ldots$
\end{itemize}

\section{Correction -- Produits de nombres relatifs}
\begin{itemize}

  \item $6\times9=54$
  \item $3\times9=27$
  \item $4\times8=32$
  \item $6\times2=12$
\end{itemize}
%%%%%%%%%%%%%%%%%%%%%%%%%%%%
\newpage
\setcounter{exo}{0}
\setcounter{section}{0}
\section{Somme algébrique}
\begin{itemize}

3
3
3
3
3
3
3
3
3
3
  \item $4+8=\ldots$
  \item $6+8=\ldots$
  \item $5+7=\ldots$
\end{itemize}

\section{Correction -- Somme algébrique}
\begin{itemize}

  \item $4+8=12$
  \item $6+8=14$
  \item $5+7=12$
\end{itemize}

\section{Produits de nombres relatifs}
\begin{itemize}

  \item $3\times7=\ldots$
  \item $9\times3=\ldots$
  \item $7\times6=\ldots$
  \item $2\times5=\ldots$
\end{itemize}

\section{Correction -- Produits de nombres relatifs}
\begin{itemize}

  \item $3\times7=21$
  \item $9\times3=27$
  \item $7\times6=42$
  \item $2\times5=10$
\end{itemize}
\end{document}