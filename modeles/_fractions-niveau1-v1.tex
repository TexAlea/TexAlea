
\exo{}

Compléter le tableau comme la première ligne ci-dessous (la zone grisée représentant la fraction).

\renewcommand{\arraystretch}{1}
\renewcommand{\tabularxcolumn}[1]{m{#1}}

{} \hfill {}
\begin{tabularx}{\columnwidth}{|c|c|c|>{\centering\arraybackslash}X|}
\hline 
Ecriture & Ecriture & Partage & Abscisse sur un axe gradué\\
décimale & fractionnaire &  &  \\
\hline
<<N[1]/M[1]>>  &
 $\dfrac{<<N[1]>>}{<<M[1]>>}$ &
  \scalebox{0.5}{\representePartage{<<N[1]>>}{<<M[1]>>}{true}} & \axeGradueFraction{0}{<<pint(N[1]/M[1]+1)>>}{<<M[1]>>}[<<N[1]>>/<<M[1]>>,{\dfrac{<<N[1]>>}{<<M[1]>>}}] \\ 
\hline
%0,4
 & $\dfrac{2}{5}$ & \scalebox{0.5}{\representePartage{6}{5}{false}} & \axeGradueFraction{0}{2}{5}{3} \\ 
\hline 
%1,333...
 & %$\dfrac{4}{3}$
 & \scalebox{0.5}{\representePartage{4}{3}{false} } & \axeGradueFraction{0}{2}{3}{3}[4/3,{\dfrac{4}{3}}] \\ 
\hline 
%1,75
 & %$\dfrac{7}{4}$
 & \scalebox{0.5}{\representePartage{7}{4}{true}}  & \axeGradueFraction{0}{2}{4}{3} \\ 
 \hline 
$-2,5$
 & %$\dfrac{-5}{2}$
 &  & \axeGradueFraction{-3}{3}{2}{1} \\ 
\hline 
\end{tabularx} 
{} \hfill {}

\exo{}


Parmi les quotients suivants, associer ensemble ceux qui sont égaux. Pour vous aider, griser les partages correspondants à chaque fraction (pas nécessairement dans l'ordre).
\quad
$\dfrac{1}{3}$, \quad $\dfrac{7}{5}$, \quad $\dfrac{4}{12}$, \quad $\dfrac{5}2$, \quad $\dfrac{10}{4}$, \quad $\dfrac{14}{10}$.

\renewcommand{\arraystretch}{2}
{} \hfill {}
\begin{tabular}{|c|c|c|c|c|c|c|}
\hline
Fraction & & & & & &  \\
\hline
Partage &
\scalebox{0.4}{\representePartage{14}{10}{false} } &
\scalebox{0.4}{\representePartage{7}{5}{false} } &
\scalebox{0.4}{\representePartage{1}{3}{false} } &
\scalebox{0.4}{\representePartage{4}{12}{false} } &
\scalebox{0.4}{\representePartage{5}{2}{false} } &
\scalebox{0.4}{\representePartage{10}{4}{false} } \\
\hline
\end{tabular} 
{} \hfill {}

\newpage

\begin{multicols}{2}



\exo{}
Recopie puis complète :
\begin{multicols}{2}
\begin{enumerate}[label=\textbf{\alph*.}]
\item $6 = \dfrac{...}{2}$
\item $7 = \dfrac{...}{2}$
\item $6 = \dfrac{...}{3}$
\item $15 = \dfrac{...}{3}$
\item $6 = \dfrac{...}{7}$
\item $10 = \dfrac{...}{7}$
\end{enumerate}
\end{multicols}




\exo{}
A l'aide de ta calculatrice, recopie puis complète par $=$ ou $\neq$:
\begin{enumerate}[label=\textbf{\alph*.}]
\begin{multicols}{2}
\item $\dfrac{1}{3} \quad ... \quad 0,33$
\item $\dfrac{19}{7} \quad ... \quad 2,714$
\item $\dfrac{15}{8} \quad ... \quad 1,875$
\item $\dfrac{3}{11} \quad ... \quad 0,27$

\end{multicols}
\end{enumerate}
\raggedcolumns

\exo{}
Recopie puis complète:
\begin{enumerate}[label=\textbf{\alph*.}]
\begin{multicols}{2}
\begin{spacing}{2}
\item $ \dfrac{...}{5} = \dfrac{10}{20}$
\item $ \dfrac{2}{3} = \dfrac{...}{27}$
\item $ \dfrac{15}{45} = \dfrac{5}{...}$
\item $ 3 = \dfrac{...}{4}$
\item $ 2,1 = \dfrac{21}{...}$
\item $ \dfrac{5}{13} = \dfrac{75}{...}$
\end{spacing}
\end{multicols}
\end{enumerate}

\exo{}
\begin{enumerate}[label=\textbf{\alph*.}]
\item Recopie les nombres suivants puis entoure en vert ceux qui sont inférieurs à 1 et en rouge ceux qui sont supérieurs à 1.\\
$\dfrac{7}{8} \quad ;   \quad\dfrac{9}{4}  \quad ;  \quad \dfrac{12}{5}  \quad ;    \quad \dfrac{634}{628}\quad ; \quad \dfrac{9}{10}\quad; \quad \dfrac{18}{8}$\\
\item Recopie les nombres suivants puis entoure ceux qui sont inférieurs à 2  en expliquant ta démarche.\\
$\dfrac{64}{21} \quad ;   \quad\dfrac{35}{18}  \quad ;  \quad \dfrac{41}{18}  \quad ;    \quad \dfrac{12}{25}\quad ; \quad \dfrac{14}{30}$\\
\end{enumerate}

\columnbreak

\exo{}
Recopie puis complète par <,> ou = :
\begin{enumerate}[label=\textbf{\alph*.}]
\begin{multicols}{2}
\begin{spacing}{2}
\item $ \dfrac{1}{3} .... 3$
\item $ \dfrac{1}{2} .... \dfrac{1}{4}$
\item $ \dfrac{2}{3} .... \dfrac{1}{9}$
\item $ \dfrac{12}{15} .... \dfrac{4}{3}$
\item $ \dfrac{7}{18} .... \dfrac{3}{9}$
\item $ \dfrac{19}{10} .... \dfrac{29}{15}$
\item 
\end{spacing}
\end{multicols}
\end{enumerate}

\exo{}
Dans les parkings la loi exige que, sur 50 places, au moins une soit réservée aux personnes handicapées. Un parking de 600 places contient 10 places pour handicapés.\\
Le gérant du parking respecte-t-il la loi?


\exo{}
\includegraphics[width=4cm]{ressources/ballon}
Le 24 octobre 2014, Alan Eustace a battu le record d'altitude en ballon en atteignant 41 419 m. Ce jour là il a effectué un saut de 900 secondes, dont 270 secondes en chute libre.
\begin{enumerate}[label=\textbf{\alph*.}]
\item Exprimer avec une fraction la proportion du temps de chute libre.
\item Cette fraction est-elle égale à:\\
\textbf{1)} un tiers ?  \quad  \textbf{2)} trois dixièmes ?  \quad \textbf{3)} deux neuvièmes ?
\end{enumerate}

\exo{}
Le musée de l'Ermitage à Saint Petersbourg est l'un des musées les plus grands du monde. Sur les 3 millions d'objet, que contient la réserve, 60 000 sont exposés. 
\begin{enumerate}[label=\textbf{\alph*.}]
\item Quelle est la proportion d'objets exposés ?
\item Le musée du Louvre à Paris expose 6,7\% de sa réserve. Lequel de ces deux musées expose la plus grande partie de sa réserve ?	
\includegraphics[width=4cm]{ressources/musee}
\end{enumerate}

\end{multicols}
