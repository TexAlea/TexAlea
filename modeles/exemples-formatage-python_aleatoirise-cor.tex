\documentclass[a4paper,11pt,fleqn]{article}
\input{preambule}
\begin{document}
\pagestyle{empty}


{Correction} \hfill {\huge Fiche d'exercices \no 1} \hfill {Classe}

\section{Somme algébrique}
\begin{itemize}

  \item $6 +3=9$
  \item $-6 -7=-13$
  \item $3 -7=-4$
\end{itemize}

\section{Produits de nombres relatifs}
\begin{itemize}

  \item $5\times6=30$
  \item $2\times7=14$
  \item $6\times2=12$
  \item $9\times(-3)=-27$
\end{itemize}

\section{Prix soldés}
\begin{itemize}

  \item Réduction : $5,40\times80\div100=\nombre{4,32}$~€\\
  Nouveau prix : $5,40-4,32=\nombre{1,08}$~€
  \item Réduction : $2.0\times30\div100=\nombre{0,60}$~€\\
  Nouveau prix : $2.0-0,60=\nombre{1,40}$~€
  \item Réduction : $1,20\times80\div100=\nombre{0,96}$~€\\
  Nouveau prix : $1,20-0,96=\nombre{0,24}$~€
\end{itemize}
%%%%%%%%%%%%%%%%%%%%%%%%%%%%
\newpage
\setcounter{exo}{0}
\setcounter{section}{0}
{Correction} \hfill {\huge Fiche d'exercices \no 2} \hfill {Classe}

\section{Somme algébrique}
\begin{itemize}

  \item $-9 -7=-16$
  \item $5 -9=-4$
  \item $-5 +2=-3$
\end{itemize}

\section{Produits de nombres relatifs}
\begin{itemize}

  \item $-4\times(-2)=8$
  \item $-4\times3=-12$
  \item $-5\times2=-10$
  \item $-6\times(-6)=36$
\end{itemize}

\section{Prix soldés}
\begin{itemize}

  \item Réduction : $2,70\times30\div100=\nombre{0,81}$~€\\
  Nouveau prix : $2,70-0,81=\nombre{1,89}$~€
  \item Réduction : $5,50\times20\div100=\nombre{1,10}$~€\\
  Nouveau prix : $5,50-1,10=\nombre{4,40}$~€
  \item Réduction : $8,30\times50\div100=\nombre{4,15}$~€\\
  Nouveau prix : $8,30-4,15=\nombre{4,15}$~€
\end{itemize}
%%%%%%%%%%%%%%%%%%%%%%%%%%%%
\newpage
\setcounter{exo}{0}
\setcounter{section}{0}
{Correction} \hfill {\huge Fiche d'exercices \no 3} \hfill {Classe}

\section{Somme algébrique}
\begin{itemize}

  \item $-4 -7=-11$
  \item $8 +8=16$
  \item $5 -3=2$
\end{itemize}

\section{Produits de nombres relatifs}
\begin{itemize}

  \item $8\times5=40$
  \item $7\times2=14$
  \item $-6\times(-7)=42$
  \item $9\times5=45$
\end{itemize}

\section{Prix soldés}
\begin{itemize}

  \item Réduction : $4,10\times60\div100=\nombre{2,46}$~€\\
  Nouveau prix : $4,10-2,46=\nombre{1,64}$~€
  \item Réduction : $9.0\times70\div100=\nombre{6,30}$~€\\
  Nouveau prix : $9.0-6,30=\nombre{2,70}$~€
  \item Réduction : $2,50\times80\div100=\nombre{2.0}$~€\\
  Nouveau prix : $2,50-2.0=\nombre{0,50}$~€
\end{itemize}
%%%%%%%%%%%%%%%%%%%%%%%%%%%%
\newpage
\setcounter{exo}{0}
\setcounter{section}{0}
{Correction} \hfill {\huge Fiche d'exercices \no 4} \hfill {Classe}

\section{Somme algébrique}
\begin{itemize}

  \item $9 -9=0$
  \item $3 +8=11$
  \item $3 +3=6$
\end{itemize}

\section{Produits de nombres relatifs}
\begin{itemize}

  \item $3\times5=15$
  \item $-7\times(-9)=63$
  \item $3\times2=6$
  \item $-3\times2=-6$
\end{itemize}

\section{Prix soldés}
\begin{itemize}

  \item Réduction : $7,10\times30\div100=\nombre{2,13}$~€\\
  Nouveau prix : $7,10-2,13=\nombre{4,97}$~€
  \item Réduction : $6.0\times70\div100=\nombre{4,20}$~€\\
  Nouveau prix : $6.0-4,20=\nombre{1,80}$~€
  \item Réduction : $7,70\times50\div100=\nombre{3,85}$~€\\
  Nouveau prix : $7,70-3,85=\nombre{3,85}$~€
\end{itemize}
%%%%%%%%%%%%%%%%%%%%%%%%%%%%
\newpage
\setcounter{exo}{0}
\setcounter{section}{0}
{Correction} \hfill {\huge Fiche d'exercices \no 5} \hfill {Classe}

\section{Somme algébrique}
\begin{itemize}

  \item $3 +8=11$
  \item $6 +4=10$
  \item $-4 -9=-13$
\end{itemize}

\section{Produits de nombres relatifs}
\begin{itemize}

  \item $8\times(-6)=-48$
  \item $-4\times3=-12$
  \item $-6\times9=-54$
  \item $-5\times(-3)=15$
\end{itemize}

\section{Prix soldés}
\begin{itemize}

  \item Réduction : $7,60\times70\div100=\nombre{5,32}$~€\\
  Nouveau prix : $7,60-5,32=\nombre{2,28}$~€
  \item Réduction : $5,60\times60\div100=\nombre{3,36}$~€\\
  Nouveau prix : $5,60-3,36=\nombre{2,24}$~€
  \item Réduction : $9,70\times40\div100=\nombre{3,88}$~€\\
  Nouveau prix : $9,70-3,88=\nombre{5,82}$~€
\end{itemize}
%%%%%%%%%%%%%%%%%%%%%%%%%%%%
\newpage
\setcounter{exo}{0}
\setcounter{section}{0}
{Correction} \hfill {\huge Fiche d'exercices \no 6} \hfill {Classe}

\section{Somme algébrique}
\begin{itemize}

  \item $5 -7=-2$
  \item $5 -4=1$
  \item $-4 +6=2$
\end{itemize}

\section{Produits de nombres relatifs}
\begin{itemize}

  \item $3\times(-4)=-12$
  \item $2\times2=4$
  \item $-8\times(-2)=16$
  \item $-9\times(-9)=81$
\end{itemize}

\section{Prix soldés}
\begin{itemize}

  \item Réduction : $4,60\times70\div100=\nombre{3,22}$~€\\
  Nouveau prix : $4,60-3,22=\nombre{1,38}$~€
  \item Réduction : $5,70\times50\div100=\nombre{2,85}$~€\\
  Nouveau prix : $5,70-2,85=\nombre{2,85}$~€
  \item Réduction : $7,40\times10\div100=\nombre{0,74}$~€\\
  Nouveau prix : $7,40-0,74=\nombre{6,66}$~€
\end{itemize}
%%%%%%%%%%%%%%%%%%%%%%%%%%%%
\newpage
\setcounter{exo}{0}
\setcounter{section}{0}
{Correction} \hfill {\huge Fiche d'exercices \no 7} \hfill {Classe}

\section{Somme algébrique}
\begin{itemize}

  \item $6 +7=13$
  \item $-3 -6=-9$
  \item $4 +3=7$
\end{itemize}

\section{Produits de nombres relatifs}
\begin{itemize}

  \item $2\times5=10$
  \item $-7\times6=-42$
  \item $9\times(-3)=-27$
  \item $-2\times(-3)=6$
\end{itemize}

\section{Prix soldés}
\begin{itemize}

  \item Réduction : $4,20\times10\div100=\nombre{0,42}$~€\\
  Nouveau prix : $4,20-0,42=\nombre{3,78}$~€
  \item Réduction : $0,80\times20\div100=\nombre{0,16}$~€\\
  Nouveau prix : $0,80-0,16=\nombre{0,64}$~€
  \item Réduction : $3,20\times20\div100=\nombre{0,64}$~€\\
  Nouveau prix : $3,20-0,64=\nombre{2,56}$~€
\end{itemize}
%%%%%%%%%%%%%%%%%%%%%%%%%%%%
\newpage
\setcounter{exo}{0}
\setcounter{section}{0}
{Correction} \hfill {\huge Fiche d'exercices \no 8} \hfill {Classe}

\section{Somme algébrique}
\begin{itemize}

  \item $-8 -4=-12$
  \item $8 -6=2$
  \item $-4 -9=-13$
\end{itemize}

\section{Produits de nombres relatifs}
\begin{itemize}

  \item $-2\times(-6)=12$
  \item $-8\times(-5)=40$
  \item $-9\times7=-63$
  \item $-7\times(-7)=49$
\end{itemize}

\section{Prix soldés}
\begin{itemize}

  \item Réduction : $8,20\times30\div100=\nombre{2,46}$~€\\
  Nouveau prix : $8,20-2,46=\nombre{5,74}$~€
  \item Réduction : $2,90\times30\div100=\nombre{0,87}$~€\\
  Nouveau prix : $2,90-0,87=\nombre{2,03}$~€
  \item Réduction : $1,40\times20\div100=\nombre{0,28}$~€\\
  Nouveau prix : $1,40-0,28=\nombre{1,12}$~€
\end{itemize}
%%%%%%%%%%%%%%%%%%%%%%%%%%%%
\newpage
\setcounter{exo}{0}
\setcounter{section}{0}
{Correction} \hfill {\huge Fiche d'exercices \no 9} \hfill {Classe}

\section{Somme algébrique}
\begin{itemize}

  \item $-4 -3=-7$
  \item $5 +7=12$
  \item $2 +9=11$
\end{itemize}

\section{Produits de nombres relatifs}
\begin{itemize}

  \item $-6\times(-9)=54$
  \item $2\times(-3)=-6$
  \item $7\times9=63$
  \item $-4\times8=-32$
\end{itemize}

\section{Prix soldés}
\begin{itemize}

  \item Réduction : $8,60\times90\div100=\nombre{7,74}$~€\\
  Nouveau prix : $8,60-7,74=\nombre{0,86}$~€
  \item Réduction : $7,10\times70\div100=\nombre{4,97}$~€\\
  Nouveau prix : $7,10-4,97=\nombre{2,13}$~€
  \item Réduction : $6,20\times70\div100=\nombre{4,34}$~€\\
  Nouveau prix : $6,20-4,34=\nombre{1,86}$~€
\end{itemize}
\end{document}