\documentclass[a4paper,11pt,fleqn]{article}		% Présentation générale et mise en page


\input{preambule}
\input{preambule-partages}
\input{preambule-axes-gradues}


%\input{preambule}
\setlength{\columnsep}{1cm}
\setlength{\columnseprule}{.5pt}


\renewcommand{\headrulewidth}{1pt}
\fancyhead[C]{\textbf{Entraînement -- Fractions et rationnels niveau 1 (v1)}} 
\fancyhead[L]{}
\fancyhead[R]{Cycle 4}

\renewcommand{\footrulewidth}{1pt}
\fancyfoot[C]{} 
\fancyfoot[L]{}
\fancyfoot[R]{Fractions et rationnels niveau 1 (v1)}

\begin{document}{Correction} \hfill {\huge Fiche d'exercices \no 1} \hfill {Classe}

\section{Somme algébrique}
\begin{itemize}

  \item $-7 +4=-3$
  \item $2 -8=-6$
  \item $-8 +9=1$
\end{itemize}

\section{Produits de nombres relatifs}
\begin{itemize}

  \item $8\times(-3)=-24$
  \item $9\times(-8)=-72$
  \item $6\times(-3)=-18$
  \item $3\times(-8)=-24$
\end{itemize}

\section{Prix soldés}
\begin{itemize}

  \item Réduction : $9,90\times40\div100=\nombre{3,96}$~€\\
  Nouveau prix : $9,90-3,96=\nombre{5,94}$~€
  \item Réduction : $8,60\times50\div100=\nombre{4,30}$~€\\
  Nouveau prix : $8,60-4,30=\nombre{4,30}$~€
  \item Réduction : $4,20\times10\div100=\nombre{0,42}$~€\\
  Nouveau prix : $4,20-0,42=\nombre{3,78}$~€
\end{itemize}
%%%%%%%%%%%%%%%%%%%%%%%%%%%%
\newpage
\setcounter{exo}{0}
\setcounter{section}{0}
{Correction} \hfill {\huge Fiche d'exercices \no 2} \hfill {Classe}

\section{Somme algébrique}
\begin{itemize}

  \item $3 +2=5$
  \item $8 +5=13$
  \item $8 +2=10$
\end{itemize}

\section{Produits de nombres relatifs}
\begin{itemize}

  \item $4\times(-7)=-28$
  \item $-3\times4=-12$
  \item $8\times(-6)=-48$
  \item $7\times2=14$
\end{itemize}

\section{Prix soldés}
\begin{itemize}

  \item Réduction : $3,20\times90\div100=\nombre{2,88}$~€\\
  Nouveau prix : $3,20-2,88=\nombre{0,32}$~€
  \item Réduction : $5,80\times50\div100=\nombre{2,90}$~€\\
  Nouveau prix : $5,80-2,90=\nombre{2,90}$~€
  \item Réduction : $4,30\times80\div100=\nombre{3,44}$~€\\
  Nouveau prix : $4,30-3,44=\nombre{0,86}$~€
\end{itemize}
%%%%%%%%%%%%%%%%%%%%%%%%%%%%
\newpage
\setcounter{exo}{0}
\setcounter{section}{0}
{Correction} \hfill {\huge Fiche d'exercices \no 3} \hfill {Classe}

\section{Somme algébrique}
\begin{itemize}

  \item $-8 -4=-12$
  \item $-3 +5=2$
  \item $9 -8=1$
\end{itemize}

\section{Produits de nombres relatifs}
\begin{itemize}

  \item $-6\times(-7)=42$
  \item $2\times(-9)=-18$
  \item $-5\times(-3)=15$
  \item $-6\times(-3)=18$
\end{itemize}

\section{Prix soldés}
\begin{itemize}

  \item Réduction : $9,60\times20\div100=\nombre{1,92}$~€\\
  Nouveau prix : $9,60-1,92=\nombre{7,68}$~€
  \item Réduction : $1,60\times50\div100=\nombre{0,80}$~€\\
  Nouveau prix : $1,60-0,80=\nombre{0,80}$~€
  \item Réduction : $8,80\times90\div100=\nombre{7,92}$~€\\
  Nouveau prix : $8,80-7,92=\nombre{0,88}$~€
\end{itemize}
%%%%%%%%%%%%%%%%%%%%%%%%%%%%
\newpage
\setcounter{exo}{0}
\setcounter{section}{0}
{Correction} \hfill {\huge Fiche d'exercices \no 4} \hfill {Classe}

\section{Somme algébrique}
\begin{itemize}

  \item $-4 +9=5$
  \item $-6 -7=-13$
  \item $-4 -9=-13$
\end{itemize}

\section{Produits de nombres relatifs}
\begin{itemize}

  \item $9\times(-9)=-81$
  \item $-2\times5=-10$
  \item $4\times5=20$
  \item $6\times5=30$
\end{itemize}

\section{Prix soldés}
\begin{itemize}

  \item Réduction : $5,30\times70\div100=\nombre{3,71}$~€\\
  Nouveau prix : $5,30-3,71=\nombre{1,59}$~€
  \item Réduction : $0,10\times10\div100=\nombre{0,01}$~€\\
  Nouveau prix : $0,10-0,01=\nombre{0,09}$~€
  \item Réduction : $1.0\times10\div100=\nombre{0,10}$~€\\
  Nouveau prix : $1.0-0,10=\nombre{0,90}$~€
\end{itemize}
%%%%%%%%%%%%%%%%%%%%%%%%%%%%
\newpage
\setcounter{exo}{0}
\setcounter{section}{0}
{Correction} \hfill {\huge Fiche d'exercices \no 5} \hfill {Classe}

\section{Somme algébrique}
\begin{itemize}

  \item $7 -8=-1$
  \item $4 -8=-4$
  \item $-9 +2=-7$
\end{itemize}

\section{Produits de nombres relatifs}
\begin{itemize}

  \item $2\times7=14$
  \item $4\times2=8$
  \item $-5\times9=-45$
  \item $5\times(-7)=-35$
\end{itemize}

\section{Prix soldés}
\begin{itemize}

  \item Réduction : $4,70\times10\div100=\nombre{0,47}$~€\\
  Nouveau prix : $4,70-0,47=\nombre{4,23}$~€
  \item Réduction : $8,30\times80\div100=\nombre{6,64}$~€\\
  Nouveau prix : $8,30-6,64=\nombre{1,66}$~€
  \item Réduction : $6,40\times20\div100=\nombre{1,28}$~€\\
  Nouveau prix : $6,40-1,28=\nombre{5,12}$~€
\end{itemize}
%%%%%%%%%%%%%%%%%%%%%%%%%%%%
\newpage
\setcounter{exo}{0}
\setcounter{section}{0}
{Correction} \hfill {\huge Fiche d'exercices \no 6} \hfill {Classe}

\section{Somme algébrique}
\begin{itemize}

  \item $2 +9=11$
  \item $-5 +9=4$
  \item $9 -9=0$
\end{itemize}

\section{Produits de nombres relatifs}
\begin{itemize}

  \item $3\times(-3)=-9$
  \item $-4\times(-8)=32$
  \item $-3\times3=-9$
  \item $5\times2=10$
\end{itemize}

\section{Prix soldés}
\begin{itemize}

  \item Réduction : $2,50\times10\div100=\nombre{0,25}$~€\\
  Nouveau prix : $2,50-0,25=\nombre{2,25}$~€
  \item Réduction : $8,90\times70\div100=\nombre{6,23}$~€\\
  Nouveau prix : $8,90-6,23=\nombre{2,67}$~€
  \item Réduction : $9,40\times30\div100=\nombre{2,82}$~€\\
  Nouveau prix : $9,40-2,82=\nombre{6,58}$~€
\end{itemize}
\end{document}