\documentclass[a4paper,11pt,fleqn]{article}
\input{preambule}
\begin{document}
\pagestyle{empty}


{Correction} \hfill {\huge Fiche d'exercices \no 1} \hfill {Classe}

\section{Somme algébrique}
\begin{itemize}

  \item $2 +7=9$
  \item $5 -6=-1$
  \item $7 -2=5$
\end{itemize}

\section{Produits de nombres relatifs}
\begin{itemize}

  \item $3\times(-5)=-15$
  \item $-9\times3=-27$
  \item $7\times3=21$
  \item $-3\times(-7)=21$
\end{itemize}

\section{Prix soldés}
\begin{itemize}

  \item Réduction : $4,90\times40\div100=\nombre{1,96}$~€\\
  Nouveau prix : $4,90-1,96=\nombre{2,94}$~€
  \item Réduction : $5,40\times30\div100=\nombre{1,62}$~€\\
  Nouveau prix : $5,40-1,62=\nombre{3,78}$~€
  \item Réduction : $3,60\times10\div100=\nombre{0,36}$~€\\
  Nouveau prix : $3,60-0,36=\nombre{3,24}$~€
\end{itemize}
%%%%%%%%%%%%%%%%%%%%%%%%%%%%
\newpage
\setcounter{exo}{0}
\setcounter{section}{0}
{Correction} \hfill {\huge Fiche d'exercices \no 2} \hfill {Classe}

\section{Somme algébrique}
\begin{itemize}

  \item $4 -7=-3$
  \item $5 +6=11$
  \item $8 -9=-1$
\end{itemize}

\section{Produits de nombres relatifs}
\begin{itemize}

  \item $7\times6=42$
  \item $7\times(-7)=-49$
  \item $8\times3=24$
  \item $-4\times4=-16$
\end{itemize}

\section{Prix soldés}
\begin{itemize}

  \item Réduction : $4.0\times70\div100=\nombre{2,80}$~€\\
  Nouveau prix : $4.0-2,80=\nombre{1,20}$~€
  \item Réduction : $2,30\times40\div100=\nombre{0,92}$~€\\
  Nouveau prix : $2,30-0,92=\nombre{1,38}$~€
  \item Réduction : $0,90\times70\div100=\nombre{0,63}$~€\\
  Nouveau prix : $0,90-0,63=\nombre{0,27}$~€
\end{itemize}
\end{document}